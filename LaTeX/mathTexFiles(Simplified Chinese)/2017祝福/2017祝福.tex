\documentclass{ctexart}\pagestyle{empty}
\author{祝xxx同学新年快乐}
\title{2017新年快乐}
\begin{document}
\maketitle
线性代数的几个解题技巧

\fbox{例1}

计算n阶行列式$$D_{n}=\left|
\begin{array}{ccccc}
x&a&a&\cdots&a\\
b&x&a&\cdots&a\\
b&b&x&\cdots&a\\
\cdots&\cdots&\cdots&\cdots&\cdots\\
b&b&b&\cdots&x\\
\end{array}
\right|$$

\fbox{解}


第n行减第n-1行,第n-1行减第n-2行,$\cdots$,得到
$$D_{n}=\left|
\begin{array}{cccccccc}
x&a&a&\cdots&a&a&a\\
b-x&x-a&0&\cdots&0&0&0\\
\cdots&\cdots&\cdots&\cdots&\cdots&\cdots&\cdots\\
0&0&0&\cdots&b-x&x-a&0\\
0&0&0&\cdots&0&b-x&x-a\\
\end{array}
\right|$$

将最后一列展开得$$D_{n}=(x-a)D_{n-1}+(-1)^{n+1}a(b-x)^{n-1}$$
即得到递推公式
\begin{equation}D_{n}=(x-a)D_{n-1}+a(x-b)^{n-1}
\end{equation}
当$a\neq b$时,
在原行列式中把$a$换成$b$,把$b$换成$a$,相当于对行列式取转置,行列式的值不变,那么便有
\begin{equation}
D_{n}=(x-b)D_{n-1}+b(x-a)^{n-1}
\end{equation}
$(1)$和$(2)$联立,消去$D_{n-1}$,即得$$D_{n}=\frac{a(x-b)^{n}-b(x-a)^{n}}{a-b}$$
当$a=b$时,容易算出,$D_{n}=(x+(n-1)a)(x-a)^{n-1}$


本题的关键在于利用对称性构造一个和递推公式对偶的式子,从而可以消去$D_{n-1}$,直接求出$D_{n}$,避免了通过递推公式求通项公式的麻烦.

\fbox{例2}

计算n阶行列式$$D_{n}=\left|
\begin{array}{ccccccc}
9&5&0&\cdots&0&0&0\\
4&9&5&\cdots&0&0&0\\
0&4&9&\cdots&0&0&0\\
\cdots&\cdots&\cdots&\cdots&\cdots&\cdots&\cdots\\
0&0&0&\cdots&9&5&0\\
0&0&0&\cdots&4&9&5\\
0&0&0&\cdots&0&4&9\\
\end{array}
\right|$$

\fbox{解}

将$D_{n}$第1列展开得$$D_{n}=9D_{n-1}-4M_{21}$$其中$M_{21}$是$a_{21}$的代数余子式,再将它按第1行展开,得$$D_{n}=9D_{n-1}-20D_{n-2}$$
这是一个数列的二阶齐次递推公式,而且已知初始项$D_{1}=9,D_{2}=61$,我们需要通过递推公式导出通项公式,这需要一些技巧.

在递推公式中将$D_{n}$替换为$x^{2}$,将$D_{n-1}$替换为$x$,将$D_{n-2}$替换为$1$
得到方程$$x^{2}=9x-20$$
解得$$x_{1}=4,x_{2}=5$$
我们计算一下这两个式子$D_{n}-x_{1}D_{n-1}$和$D_{n}-x_{2}D_{n-1}$
,发现有$$D_{n}-4D_{n-1}=5(D_{n-1}-4D_{n-2})$$
$$D_{n}-5D_{n-1}=4(D_{n-1}-5D_{n-2})$$
也就是说,数列\{$D_{n}-4D_{n-1}$\}和\{$D_{n}-5D_{n-1}$\}都是等比数列,是不是很神奇?
(事实上,具体是什么原理我也没弄明白)

因此根据$D_{1}=9,D_{2}=61$很容易求出
$$D_{n}-4D_{n-1}=5^{n}$$
$$D_{n}-5D_{n-1}=4^{n}$$
消去$D_{n-1}$就得到$$D_{n}=5^{n+1}-4^{n+1}$$

下面我们来看一个线性方程组问题的栗子(注意,这不是错别字,这是强行卖萌)

\fbox{例3}

$a,b$取何值时以下方程组有唯一解,无穷多解,无解?
$$
\left\{
\begin{array}{l}
ax_{1}+bx_{2}+2x_{3}=1\\
ax_{1}+(2b-1)x_{2}+3x_{3}=1\\
ax_{1}+bx_{2}+(b+3)x_{3}=1\\
\end{array}
\right.
$$


\fbox{解}

由于系数矩阵是方阵,我们可以用Cramer法则找出唯一解的情况,这样可以减少计算量.(如果系数矩阵不是方阵则只能通过增广矩阵的行变换求解)

系数矩阵的行列式$D_{n}=a(b-1)(b+1)$

故当$a\neq0,b\neq\pm1$时有唯一解
$\left(
\begin{array}{c}
x_{1}\\
x_{2}\\
x_{3}\\
\end{array}
\right)=\left(
\begin{array}{c}
\frac{5-b}{a(b+1)}\\
-\frac{2}{b+1}\\
\frac{2(b-1)}{b+1}\\
\end{array}
\right)$

当$D=0$时,下面进行分类讨论

(i)当$a=0$时,对增广矩阵化简得
$$\left(
\begin{array}{cccc}
0&1&1&1\\
0&0&2-b&1-b\\
0&0&b+1&2b-2\\
\end{array}
\right)$$
要使方程组有解,第2行和第3行必成比例,否则无解

即$b=1$或$b=5$时有无穷多解,$b\neq1$且$b\neq5$时无解

当$b=1$时原方程组的解为$\left(
\begin{array}{c}
x_{1}\\
x_{2}\\
x_{3}\\
\end{array}
\right)=\left(
\begin{array}{c}
0\\
1\\
0\\
\end{array}
\right)+c\left(
\begin{array}{c}
1\\
0\\
0\\
\end{array}
\right)$



当$b=5$时原方程组的解为$\left(
\begin{array}{c}
x_{1}\\
x_{2}\\
x_{3}\\
\end{array}
\right)=\left(
\begin{array}{c}
0\\
-\frac{1}{3}\\
\frac{4}{3}\\
\end{array}
\right)+c\left(
\begin{array}{c}
1\\
0\\
0\\
\end{array}
\right)$

其中c为任意常数


(ii)当$a\neq0,b=1$时,同理可得原方程有无穷多解$$\left(
\begin{array}{c}
x_{1}\\
x_{2}\\
x_{3}\\
\end{array}
\right)=\left(
\begin{array}{c}
\frac{1}{a}\\
0\\
0\\
\end{array}
\right)+c\left(
\begin{array}{c}
-\frac{1}{a}\\
1\\
0\\
\end{array}
\right)$$


其中c为任意常数


(iii)当$a\neq0,b=-1$时,同理可得原方程组无解

综合上面的讨论,得

$a\neq0,b\neq\pm1$时有唯一解

$a=0,b=1$时或$a=0,b=5$时或$a\neq0,b=1$时有无穷多解

$a=0,b\neq1$且$b\neq5$时或$a\neq0,b=-1$时无解



\end{document}




