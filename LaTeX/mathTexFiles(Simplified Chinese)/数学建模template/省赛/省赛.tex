\documentclass{ctexart}
\pagestyle{plain}
\usepackage{makecell,rotating,multirow,diagbox}
\usepackage{indentfirst}
\usepackage{tabularx}
\usepackage{amsmath}
\date{}
\title{试卷出题方案的研究及应用}
\begin{document}
\maketitle
\section*{摘要}
考试是对学生学习效果的检验,制定合理的出题方案是拟定考题的前提。我们利用已知的七个不同准则对拟考点的属性状况分别进行定性和定量分析,采用了Matlab软件编程,对所给问题分别给出了数学模型及处理方案。

针对问题一,我们首先提出了三种方法,分别是相似度分析法、定性和定量分析相结合以及层次分析法,对应每种方法建立了数学模型,并对这些方法进行改良和创新。我们对三种方法的优缺点从计算量、主客观性等方面进行了讨论,给出了客观且具有综合性的评价。


针对问题二,我们根据三种方法提出了具体的方案,应用了统计分析和Matlab编程求解,得到了相应的拟考点,并从计算量、实施性等方面进行了对比说明。



针对问题三,根据不同方案的特点,得到最终的出题方案。

针对问题四,我们根据方案三的方案,从题库中抽取题目,拟定试卷。

针对问题五,对试卷从难度、区分度、知识点覆盖率进行综合评价。

最终我们对于所建立的数学模型的优缺点作出了评价。
\section*{关键词}
拟考点\  相似度分析\ 定性分析\ 层次分析法\  综合评价

\section{问题重述}
一套切合实际、合理的试卷,不仅能够全面、客观的考核学生的学习效果,也能够考察教师的教学效果。制定客观、合理的出题方案是拟定考题的前提。一般来说,制定出题方案时,主要对知识点从“基础性、重要性、难易度、启发性、延伸性、应用性、关联性”这七个准则进行综合考察。

对于期末考题而言,不妨这七个准则按照重要程度排序为:

基础性、关联性、重要性、应用性、难易度、延伸性、启发性

现在有一门课程需要制定期末考试出题方案,课程共有N章内容,每一章有若干节内容,每一节又有若干个拟考点。
不同章、不同节、不同拟考点对于七个准则的属性状况均不同。一般地,这些属性状况既可以用0—10之间的11个等级定量刻划,又可以用不同的优劣等级来定性描述。
试卷的总体难易程度最好控制在中等或中等稍偏下水平。

问题1:对于制定出题方案,请至少研究三种不同的方法,并分析、说明每种方法的特点及不同方法之间的差异。

问题2:通过广泛调研、咨询,并利用问题1给出的各种方法,分别制定高等数学(上)的期末出题方案,并对这些方案加以对比、说明。

问题3:综合不同方案,制定高等数学(上)的期末出题方案。

问题4:请根据问题3给出的出题方案,拟定两套期末考题,并给出答案。(考试时长为120分钟)

问题5:对所拟定的考题进行综合评价,说明其合理性。

\section{定义、假设和符号说明}
\subsection{总体的基本假设}

1.假设每个拟知识点的七个准则的属性值已知

2.假设每套期末试卷的题型固定

3.根据拟考点在多份试卷中出现的概率可以衡量一个拟考点的重要程度
\subsection{定义及符号说明}
$r_{ij}$——相似度

$P$——整卷难度

$P_{i}$ ——某试题难度

$X_{i}$——某试题满分值

$W$——试卷满分值

$D$——区分度

$X_{H}$:高分组某试题的平均分

$X_{L}$:低分组某试题的平均分

$X_{T}$:该题的满分


\section{问题分析和模型建立}
\subsection{问题分析}
本题是根据拟考点的七个准则进行评估,制定方法,拟定期末考试出题方案,最后给出具体的出题题目并进行综合评价。

其中问题一要解决的是找到至少三种合适的方法。为了使试卷出的更加合理,在选取方法时应该考虑定性和定量分析,难点在于很难有一个合适的数学模型直接与要解决的问题契合,所以要对模型进行一些改良与创新。制定方法后要分析说明不同方法的特点并比较。

问题二是对针对不同方法提出具体的方案并进行说明比较,需要算法和数据作为支撑,以得到具体的方案。

问题三是综合方案,选择一个最优的方法去制定出题方案,需要进行综合考量。

问题四是具体应用,把从第三问得到的拟考点进行整合,拟定期末试题,同时要把试卷的总体难易程度控制在中等或中等稍偏下水平。

问题五是一个综合评价的问题,需要一些指标对试卷的合理性进行评价。而对于试卷的合理性可以从难度、知识点的覆盖情况、区分度等进行测评。
\subsection{模型建立及问题求解}
\subsubsection{问题一的求解}

\textbf{方法一:相似度分析法}

本方法通过简单随机抽样的方法抽取历年各个高校相同或相似的期末考题,再根据抽取的样本计算出每个拟考点$x_{i}$在试卷中出现的概率$P(x_{i})$,$P(x_{i})$越大,我们就认为$x_{i}$这个拟考点是越重要的,从而为出题方案提供一定的参考。
为了衡量两个拟考点之间的相似程度,利用每个拟考点的7个属性,设每个考点$x_{i}$对应的7个属性值为$\overrightarrow{x_{i}}=(a_{i1},a_{i2},...,a_{i7})$

定义相似度$r_{ij}$代表第$x_{i}$个考点和第$x_{j}$个考点的相似度,计算方法为:
$$r_{ij}=\frac{1}{\sqrt{\sum\limits_{k=1}^{7}(a_{ik}-a_{jk})^2}})$$
$r_{ij}$越大代表$x_{i}$和$x_{j}$越相似。

通过对抽样的试卷进行分析,利用上述方法得到每个拟考点出现的概率$P(x_{i})$进行排序,选取前q个概率最高的拟考点,再对这q个拟考点进行相似度分析。假设当两个拟考点a,b的相似度大于某个值M时,根据$P(a)$ 和$P(b)$舍弃概率较小的拟考点。通过这种方法选择出的拟考点出现概率比较高,同时又不十分相似,所以既可以避免一份试卷中出现知识点重复考察的现象,同时又可以保证该考点的考察是较重要的。

\textbf{特点}:借助现有的考试题,使得出题方案所包括的考点集中在重要程度较高的拟考点上,同时避免了知识点重复考察的情况,保证了试卷的质量。


\textbf{方法二:}

首先我们对每个拟考点的7个属性进行定性分析,用0到10这11个等级刻画,设所有的拟考点有N个,一份试卷含有n个拟考点(n<N),记这n 个拟考点为$x_{i}(i=1,2,…,n)$, 每个拟考点对应的属性值为$\overrightarrow{x_{i}}=(a_{i1},a_{i2},...,a_{i7})$, $a_{ij}\in\{0,1,...,10\}$

定义一份试卷对应的7个属性为$$(A_{1},A_{2},…,A_{7})=(\sum\limits_{i=1}^{n}a_{i1}, \sum\limits_{i=1}^{n}a_{i2},…, \sum\limits_{i=1}^{n}a_{i7})$$我们抽取各个高校的高质量的期末考试试题,通过分析试题中含有的知识点,再根据知识点对应的7个属性的值,得到每份试题的$A_{1}$,…,$A_{7}$的值。进而计算出优秀试题中每个$A_{i}$ 的大致范围,记为$(A_{i}-b_{i},A_{i}+c_{i})$。那么出题方案所包括的n 个拟考点的7个属性应满足
$$A_{1}-b_{1}<\sum\limits_{i=1}^{n}a_{i1}<A_{1}+c_{1}$$

$$A_{2}-b_{2}<\sum\limits_{i=1}^{n}a_{i2}<A_{2}+c_{2}$$

$$\cdots  \cdots$$

$$A_{7}-b_{7}<\sum\limits_{i=1}^{n}a_{i7}<A_{7}+c_{7}$$


由于每个考点的7个属性已知,只需要从N个考点中选取n个考点,对n个考点的7个属性总和进行计算,如果满足上述7个不等式,即可视为选取的这n 个拟考点是合理的。

\textbf{特点}:定性分析和定量分析结合,可以找到满足要求的所有拟考点组合,在保证试卷质量的前提下相对增加了知识点的丰富性。不足之处是计算量稍大。


\textbf{方法三:层次分析法}

我们参照层次分析法的思想,通过调研得到相关数据并计算数据的一致性,在数据具有较好一致性的前提下,我们可计算得到各准则在同一参考系下(拟考点)的量化的影响程度。具体模型如下:

不妨设我们一共有t个拟考点,依次编号为1,2,...,t,对于第i个拟考点,我们设此拟考点的基础性、关联性、重要性、应用性、难易度、延伸性、启发性的属性值分别为 $x_{i1}$,$x_{i2}$,...,$x_{i7}$,有了这方面的数据之后,我们需要利用这些拟考点的属性值来衡量此拟考点的“质量”,即此拟考点多大程度上适合作为考题涵盖的知识点。

为刻画拟考点的“质量”,我们引入刻画一个考点质量的函数$$f(x_{1},x_{2},...,x_{7})$$

$f$的值越大代表一个考点的质量越高,越需要被考察。最简单的$f$的模型即是线性函数,在此方法中我们也是以线性函数来研究。

设$f=\sum\limits_{j=1}^{7}p_{j}x_{ij}$,其中$p_{1}$,$p_{2}$,...,$p_{7}$依次为上述七种准则的权重,由此七种准则的定性比较我们可知$p_{1}$>$p_{2}$>$p_{3}$>$p_{4}$>$p_{5}$>$p_{6}$>$p_{7}$下面我们利用层次分析法的主要思想,来确定$p_{i}$的值。

为获取$p_{i}$的具体权重,我们需要充分利用人的主观判断,主观判断虽然带有一定的模糊性,但在数据量较为缺乏时,加以科学地利用,不失为一种优秀的方法,层次分析法就充分利用了这一点。为此,我们来利用主观因素来确定判断矩阵$A=(a_{ij})_{7\times7}$,设前述七准则分别为$c_{1}$ ,$c_{2}$,...,$c_{7}$我们令$a_{ij}$为$c_{i}$和$c_{j}$对拟考点的“质量”影响之比。接下来为保证主观数据的合理性,我们需要计算判断矩阵的一致性指标CI和一致性比例CR,在CR<0.10时即认为判断矩阵的一致性是可以接受的,即此主观判断是合理的,在此前提下,我们计算出A 相对于最大特征值$\lambda_{max}$的特征向量W,经归一化之后即是前述七准则对拟考点的“质量”的量化之比,即所找的各$p_{i}$。

在确定$p_{i}$之后,即可利用拟考点的各准则属性值来计算拟考点的“质量”,为制定良好的出题方案,我们只需选取部分“质量”最佳的拟考点作为试卷需涵盖的考点即可。

\textbf{特点}:利用层次分析法计算权重,充分利用了主观数据的同时,也避免了主观因素有可能带来的较大误差,提高了主观判断的准确度。因此这种判断准则权重的方法是比较科学的。

\textbf{三种方法的差异比较}:第一种方法从拟考点的相似度和题目出现的概率出发,第二种方法从每个考点的属性本身出发,第三种方法从权重出发。从定性与定量角度看,第一种和第三种方法采用了定量分析,第二种方法采用了定性与定量相结合的方法。从主观与客观性的角度,第一种和第二种方法较为客观,第三种方法较为主观。从试卷质量来考察的话,第一种方法与第二种方法更好地保障了出题质量。总体来说第一种方法的科学性最强,第二种方法的计算量较大,第三种方法的主观性最大。

\subsubsection{问题二的求解}

\textbf{方案一:}

我们通过调查,咨询不同水平的学生确定了每个拟考点7个属性的相对值。(利用0-10这11个等级刻画)

0代表极其不重要;

2代表不重要;

4代表比较不重要;

6代表比较重要;

8代表重要;

10代表极其重要;

1,3,5,7,9分别代表重要程度介于0,2,4,6,8,10之间。

每个拟考点的7个属性见附录1

各个拟考点在试卷中出现的概率统计图见附录2

选择出现概率大于等于0.3的拟考点进行相似度分析

相似度计算利用附录3的代码

我们认为当两个考点的相似度大于或等于0.5时,这两个考点重复考察会降低试卷的质量。

经过计算,我们发现考点\emph{‘利用换元积分法计算定积分’} 和\emph{‘利用第一类换元积分法计算不定积分’}过于相似,因此我们舍去后者;考点\emph{‘利用“第2节”的方法求函数的二阶导数’},\emph{‘求隐函数的一阶导数’},\emph{‘求参数方程所确定函数的一阶导数’}和\emph{‘求参数方程所确定函数的二阶导数’}这四个考点过于相似,因此我们只选取考点\emph{‘利用“第2节”的方法求函数的二阶导数’}和\emph{‘求参数方程所确定函数的一阶导数’}。由此确定了出题时所要包含的考点为:

考点1:利用洛必达法则求极限

考点2:计算无穷限的反常积分

考点3:利用等价无穷小代换求极限

考点4:利用导数运算法则及基本求导公式直接求函数的导数

考点5:利用微分中值定理证明问题

考点6:计算平面图形的面积

考点7:利用换元积分法计算定积分

考点8:利用左、右导数的定义判断函数在某些点的可导性

考点9:利用运算法则求函数极限

考点10:综合利用最值性、有界性、界值性证明问题

考点11:利用“第2节”的方法求函数的二阶导数

考点12:求参数方程所确定函数的一阶导数

考点13:描绘函数图形

考点14:计算无界函数的反常积分

考点15:利用单侧极限的定义、性质证明函数极限存在与否

考点16:利用夹逼准则求极限

考点17:利用泰勒公式求极限

考点18:利用函数的单调性证明不等式

考点19:解决实际问题(第3章第5节)

考点20:利用基本积分表直接计算不定积分

考点21:利用微积分基本公式计算定积分

\textbf{方案二:}

根据问题2  方法1 的调查数据,我们可以确定各个考点的7个属性的定量刻画,和方法1同理,数字越大代表该考点所具有的性质越高。随机抽取10 套试卷,分别计算每套试卷的7个属性,有

\begin{tabular}{|c|*{10}{c}|}
\hline
\diagbox{属性值}{试卷编号}&1&2&3&4&5&6&7&8&9&10
\\
\hline
\thead{$A_{1}$}&88&71&70&67&75&80&83&74&86&92\\
\hline
\thead{$A_{2}$}&78&62&56&63&64&73&74&64&77&72\\
\hline
\thead{$A_{3}$}&87&71&71&70&72&70&71&70&75&82\\
\hline
\thead{$A_{4}$}&78&61&63&60&65&67&63&67&69&82\\
\hline
\thead{$A_{5}$}&66&64&44&62&67&64&67&63&65&63\\
\hline
\thead{$A_{6}$}&64&55&47&52&54&51&57&55&59&64\\
\hline
\thead{$A_{7}$}&51&52&41&38&47&39&46&46&61&53\\
\hline
\end{tabular}

由上述表格,我们对每一行除去最大值和最小值,得到$A_{i}$的最佳范围和平均数:

70<$A_{1}$<88       \qquad       $\overline{A_{1}}$=78.735

62<$A_{2}$<77       \qquad     $\overline{A_{2}}$=68.625

70<$A_{3}$<82       \qquad       $\overline{A_{3}}$=78.75

61<$A_{4}$<78       \qquad       $\overline{A_{4}}$=66.625

62<$A_{5}$<67       \qquad       $\overline{A_{5}}$=64.25

51<$A_{6}$<64       \qquad       $\overline{A_{6}}$=55.875

39<$A_{7}$<53       \qquad       $\overline{A_{7}}$=46.875

即每一套试题选出的拟考点应满足:

          $$70<\sum\limits_{i=1}^{n}x_{i1}<88$$
          $$62<\sum\limits_{i=1}^{n}x_{i2}<77 $$
         $$\cdots\cdots$$
          $$41<\sum\limits_{i=1}^{n}x_{i7}<53$$

并且越接近平均数越好。

我们只需要对这67个考点进行组合,假设一份试卷包含17个左右的拟考点,即需要验证$C_{67}^{17}$种,约为$3\cdot10^{14}$,这是一个非常大的数字,我们认为没有必要全部找出符合条件的考点组合,只需要根据教学经验和考试经验大致确定考点,再依据上述不等式组对出题所选择的考点进行约束即可,并且尽可能接近平均数$A_{i}$。

\textbf{方案三:}

如此方法前述,我们需要制定七个准则分别对拟考点“质量”的影响比例的判断矩阵,为消除个人主观因素的影响,我们访问了不同的学生,制作出附录中\emph{“A.层次分析”}表格,此表格对应矩阵记为$A$。

我们在调研的同时也统计了不同水平的学生对各个拟考点的准则属性值的判断,并经整理后制作出了附录\emph{“B.高数知识点等级评价”}

首先,我们对判断矩阵$A$的一致性进行检验,利用MATLAB求得A的一致性指标CI=0.0882与一致性比例CR=0.0668,由于CR<0.10,此判断矩阵的一致性是可以接受的。

利用附录4中的代码,我们可以求得A相对于最大特征值的特征向量W,W的各个分量的比值即为各个$p_{i}$的比例,再令特征向量W的模长为1,可得到各个$p_{i}$的值为

$p_{1}=0.857$,

$p_{2}=0.362$,

$p_{3}=0.293$,

$p_{4}=0.175$,

$p_{5}=0.108$,

$p_{6}=0.063$,

$p_{7}=0.042$,

确定出$p_{i}$后即可计算出每个考点的函数值$f(x_{1},...,x_{7})$,然后取前20个函数值最高的考点即可,最终确定的考点是:

考点1:利用微积分基本公式计算定积分

考点2:利用等价无穷小代换求极限

考点3:利用定积分性质及微积分基本公式计算函数极限

考点4:求参数方程所确定函数的一阶导数

考点5:求参数方程所确定函数的二阶导数

考点6:利用反函数求导法则求函数的导数

考点7:利用定积分的性质证明问题

考点8:利用定义证明或验证函数极限

考点9:求函数的微分

考点10:利用导数在变化率问题中的应用解决实际问题

考点11:利用分部积分法计算定积分

考点12:利用“函数的求导法则”的方法求函数的二阶导数
	
考点13:求隐函数的一阶导数

考点14:利用换元积分法计算定积分

考点15:利用洛必达法则求极限

考点16:求隐函数的二阶导数

考点17:利用左、右导数的定义判断函数在某些点的可导性

考点18:利用“函数的求导法则”的方法求函数的三阶导数

考点19:函数的微分解决实际问题

考点20:利用复合函数求导法则求函数的导数

\textbf{对比说明:}

从模型的求解结果,即得到的具体方案可以看出,方案一和方案三得到的都是具体的考点,由于方案二得到具体考点花费的计算量太大,我们只能得到每份纸卷的考点需要满足的约束条件。方案一考虑的重点是考点出现的概率和相似度,方案二考虑的重点是每个属性的累加值,方案三考虑的重点是权重。方案一结合概率使得所得到的考点更加科学,同时将相似度较高的知识点排除,提高了试题的质量。而方案三的主观性较强,需要满足一致性保证相对的客观性。所以方案一相比下更为合理。
\subsubsection{问题三的求解}

为了在保证考点质量的前提下尽可能覆盖多的考点,我们选取17个考点作为试卷的主要考点。

首先,我们将方案一的21个考点与方案三的20个考点对比,得到7个重合的知识点,我们认为重合的知识点非常重要,将它们视为必考。由于方案一是建立在概率的基础上进行的操作,而方案三具有一定的主观性,所以将方案三中与方案一不重合的13个知识点排除掉,在方案一中与方案三不重合的14 个考点中抽取10 个,这样就得到了一组含有17个考点的考点组合。

对所有这样的的考点组合计算各个组合的7个属性$A_{i}$,我们认为对于一组考点的7个属性$A_{i}$来说,它们的比例越接近方法二中$A_{i}$ 平均数$\overline{A_{i}}$, 那么这些知识点就是越好的,计算的代码见附录5。

选取一组$A_{i}$ 最接近的$\overline{A_{i}}$的考点,得到最终的出题方案:

考点1:利用洛必达法则求极限

考点2:利用微积分基本公式计算定积分

考点3:利用等价无穷小代换求极限

考点4:求参数方程所确定函数的一阶导数

考点5:利用“第2节”的方法求函数的二阶导数

考点6:利用左、右导数的定义判断函数在某些点的可导性

考点7:利用换元积分法计算定积分

考点8:计算无穷限的反常积分

考点9:利用运算法则求函数的极限

考点10:计算平面图形的面积

考点11:利用微分中值定理证明问题

考点12:计算无界函数的反常积分

考点13:利用单侧极限的定义性质证明函数极限存在与否

考点14:利用夹逼准则求极限

考点15:利用泰勒公式求极限

考点16:利用函数的单调性证明不等式

考点17:利用基本积分表直接计算不定积分


\subsubsection{问题四的求解}

\textbf{我们拟定的题目为:}

\textbf{试卷一:}
\begin{center}
\textbf{XX大学期末试卷(A)\\
20xx —20xx	学年第一学期\\}
\end{center}
\textbf{一.填空题}

1.计算$\lim\limits_{x\rightarrow +\infty}x[ln(1+x)-lnx]=$\underline{\hspace{3em}}

2.若$f(x)=cosx\cdot e^{sinx}$,则$f'(x)=$\underline{\hspace{3em}}

3.计算$\lim\limits_{n\rightarrow +\infty}\frac{2^{n}}{n!}=$\underline{\hspace{3em}}

4.函数$y=\frac{2^{\frac{1}{x}-1}}{2^{\frac{1}{x}}+1}$的间断点的类型是\underline{\hspace{3em}}

5.已知$f'(1)=2$,求$\lim\limits_{x\rightarrow 0}\frac{f(1+x)-f(1-x)}{x}=$\underline{\hspace{3em}}

6.$y=arctan\frac{y}{x}$,求$\frac{dy}{dx}=$\underline{\hspace{3em}}

7.$\frac{1}{\sqrt{1+e^{2x}}}$的一个原函数为\underline{\hspace{3em}}

8.计算$\int_{-1}^{1}(x^{2}sinx+\frac{1}{1+x^{2}})dx=$\underline{\hspace{3em}}

\textbf{二.计算题}

9.求极限$\lim\limits_{n\rightarrow +\infty}(\frac{1}{n^{2}+n+1}+\frac{2}{n^{2}+n+2}+\cdots +\frac{n}{n^{2}+n+n})$

10.$y=\sqrt[5]{\frac{x(x+1)}{x-1}}$,求$\frac{dy}{dx}$

11.计算定积分$\int_{0}^{\frac{1}{2}}\frac{x^{2}}{\sqrt{1-x^{2}}}dx$

\textbf{三.证明题}

12.设$f(x)$和$F(x)$分别在[a,b]上连续,(a,b)上可导,$f(x)$非零,$f(b)=1$,极限$\lim\limits_{x\rightarrow a^{+}}\frac{f(x)}{x-a}$存在,且满足$F'(x)=f(x)$,试证明在(a,b)内至少存在相异的两个点$\xi$和$\zeta$,使得
$$(\xi-a)f'(\zeta)=f'(\xi)[F(b)-F(a)]$$


\begin{center}
\textbf{参考答案}
\end{center}
1.1 \qquad 2.$-sinx\cdot e^{sinx}+cos^{2}x\cdot e^{sinx}$  \qquad 3.0  \qquad 4.第一类间断点(跳跃间断点)  \qquad  5.4
\qquad  6.$\frac{y}{x}$  \qquad  7.$\frac{1}{2}ln(\frac{\sqrt{1+e^{2x}}-1}{\sqrt{1+e^{2x}}+1})$  \qquad  8.$\frac{\pi}{2}$

9.$\frac{1}{2}$  \qquad  10.$\frac{1}{5}(\frac{1}{x}+\frac{1}{x+1}-\frac{1}{x-1})\sqrt[5]{\frac{x(x+1)}{x-1}}$  \qquad
11.$\frac{\pi}{12}-\frac{\sqrt{3}}{8}$

12.由极限的存在性知$\lim\limits_{x\rightarrow a^{+}}f(x)=0$,又f(x)在x=a处右连续知$f(a)=\lim\limits_{x\rightarrow a^{+}}f(x)=0$,对f与F分别应用柯西中值定理得:
$$\frac{1-0}{F(b)-F(a)}=\frac{f(b)-f(a)}{F(b)-F(a)}=\frac{f'(\xi)}{F'(\xi)}=\frac{f'(\xi)}{f(\xi)},a<\xi<b$$
函数$f(x)$在$[a,\xi]$上应用拉格朗日中值定理得:
$$f(\xi)=f(\xi)-f(a)=f'(\zeta)(\xi-a),a<\zeta<\xi$$
两式综合即证

\textbf{试卷二:}
\begin{center}
\textbf{XX大学期末试卷(A)\\
20xx —20xx	学年第一学期\\}
\end{center}
\textbf{一.填空题}

1.计算$\lim\limits_{x\rightarrow +\infty}\frac{(sinx)\cdot ln(1+2x)}{1-cos2x}=$\underline{\hspace{3em}}

2.设$f(x)=\lim\limits_{t\rightarrow +\infty}x(1+\frac{1}{t})^{4xt}$,求$f'(x)$=\underline{\hspace{3em}}

3.判断$f(x)=\frac{1}{1+e^{\frac{1}{x}}}$的间断点类型\underline{\hspace{3em}}

A.可去间断点  \qquad  B.跳跃间断点  \qquad  C.第二类间断点

4.计算$\int_{-1}^{1}x^{2}(1+\sqrt{1+x^{2}}six)dx$=\underline{\hspace{3em}}

5.方程$4arctanx-x+\frac{4\pi}{3}-\sqrt{3}$的实根个数为\underline{\hspace{3em}}

6.设$D$是由曲线$xy+1=0$,$x+y=0$,$y=2$围成的有界区域,则$D$的面积是\underline{\hspace{3em}}

7.设$y=y(x)$是由方程$xy+e^{y}=x+1$确定的隐函数,则$\frac{d^{2}y}{dx^{2}}\mid_{x=0}=$\underline{\hspace{3em}}

8.已知$f(0)=f'(0)=1$,求$n[f(\frac{2}{n})-1]=$\underline{\hspace{3em}}

\textbf{二.计算题}

9.求极限$\lim\limits_{x\rightarrow 0}\frac{e^{x^{2}}-e^{2-2cosx}}{x^{4}}$

10.已知参数方程$$\left\{
\begin{aligned}
x&=&\ t+cos2t\\
y&=&\ 2t+sin2t\\
\end{aligned}
\right.
$$
计算$\frac{d^{2}y}{dx^{2}}\mid_{x=0}$

11.计算$\int_{a}^{b}\frac{dx}{\sqrt{(x-a)(b-x)}},(a<b)$

\textbf{三.证明题}

12.设函数$f(x)$在[0,1]上连续,(0,1)上可导,且$f(0)=f(1)=0,f(\frac{1}{2})=1$,试证明

(1)存在$\zeta\in(\frac{1}{2},1)$,使得$f(\zeta)=\zeta$

(2)对任意实数$\lambda$,必存在$\xi\in(0,\zeta)$,使得$f'(\xi)-\lambda[f(\xi)-\xi]=1$

\begin{center}
\textbf{参考答案}
\end{center}

1.1  \qquad 2.$e^{4x}+4xe^{4x}$  \qquad  3.B  \qquad  4.$\frac{2}{3}$	\qquad	5.  2	\qquad
6.  2  \qquad   7.  -3  \qquad  8.   2	 	

9.$\frac{1}{12}$
\qquad 10.16
\qquad  11.$\pi$

12.
(1)令$F(x)=f(x)-x$,由题设知$F(x)$在[0,1]连续,又$F(\frac{1}{2})=f(\frac{1}{2})-\frac{1}{2}>0$,$F(1)=f(1)-1<0$,由连续函数零点定理知存在$\zeta\in(\frac{1}{2},1)$,使$F(\zeta)=0$,即$f(\zeta)=\zeta$

(2)令$\varphi(x)=e^{-\lambda x}[f(x)-x]$,则$\varphi(x)$在$[0,\zeta]$上满足罗尔定理的条件,所以存在$\xi\in(0,\zeta)$使得
$\varphi'(x)=0$,即$(f'(\xi)-1)-\lambda(f\xi)-\xi)=0$,即$f'(\xi)-\lambda[f(\xi)-\xi]=1$

\subsubsection{问题五的求解}

我们对所出试题进行了广泛调研以及统计分析,综合以下三个指标来评价

1.试题的难度

难度是指试题的难易程度,一般用字母P表示,P越大表示试题越简单,P越小表示试题越难。试题的难度决定了整套试卷的难度以及考试分数的分布。

在每道题难度预估的基础上,可利用下列公式计算整套试卷的预估难度[1]
     $$p=\frac{\sum\limits_{i=1}^{n}x_{i}p_{i}}{W}$$
	
 P:整卷难度    $P_{i}$:某试题难度    $X_{i}$:某试题满分值     W:试卷满分值
整套试卷的难度控制在0.5左右。

2.试题的区分度

区分度是指试题对被试者情况的分辨能力的大小。[2]区分度指标的评价: $-1.00\leq D\leq+1.00$,区分度指数越高,试题的区分度就越强。一般认为,区分度指数高于0.3,试题便可以被接受。
                $$D=\frac{X_{H}-X_{L}}{X}$$
$X_{H}$:高分组某试题的平均分,$X_{L}$:低分组某试题的平均分,X满:该题的满分。

3.知识点覆盖率

知识点覆盖率是指每套试卷所出现的知识点占所有知识点的比重,体现了考题的内容是否足够全面。

\textbf{第一套试卷}

\begin{tabular}{|c|*{4}{c}| }
\hline
题号	&知识点	&分值	&难度&	区分度\\
\hline
1	&利用等价无穷小代换求极限&	7	&0.3&	0.4\\
\hline
2	&利用导数运算法则及基本求导公式直接求函数的导数&	7	&0.3&	0.2\\
\hline
3	&利用夹逼准则求极限	&7&	0.3&	0.3\\
\hline
4&	利用运算法则求函数极限&	7	&0.5&	0.4\\
\hline
5	&利用单侧极限的定义、性质证明函数极限存在与否&	7&	0.5	&0.5\\
\hline
6&	利用等价无穷小代换求极限&	7	&0.7&	0.7\\
\hline
7	&利用换元积分法计算定积分&	7	&0.6&	0.6\\
\hline
8	&利用微积分基本公式计算定积分&	7	&0.6	&0.5\\
\hline
9	&利用夹逼准则求极限	&11	&0.4&	0.5\\
\hline
10	&利用复合函数求导法则求函数的导数	&11	&0.5	&0.5\\
\hline
11	&利用分部积分法计算不定积分&	12	&0.6	&0.7\\
\hline
12	&利用微分中值定理证明问题	&10&	0.7&	0.9\\
\hline

\end{tabular}


试卷总难度0.5

试卷总区分度8.91

知识点覆盖率64.7\%

\textbf{第二套试卷}

\begin{tabular}{|c|*{4}{c}| }
\hline
题号&	知识点		&分值&	难度&	区分度\\
\hline
1&	利用等价无穷小代换求极限	&	7	&0.3&	0.2\\
\hline
2	&利用重要极限求极限	&	7	&0.4	&0.4\\
\hline
3	&间断点的类型	&	7	&0.6	&0.3\\
\hline
4	&利用微积分基本公式计算定积分	&	7	&0.4&	0.3\\
\hline
5	&综合利用最值性、有界性、界值性证明问题		&7&	0.7&	0.4\\
\hline
6	&计算平面图形的面积	&	7	&0.3&	0.1\\
\hline
7	&求隐函数的二阶导数	&	7	&0.9	&0.8\\
\hline
8	&利用洛必达法则求极限	&	7	&0.8&	0.5\\
\hline
9	&利用洛必达法则求极限	&	11	&0.3&	0.4\\
\hline
10	&求参数方程所确定函数的二阶导数	&	11	&0.5	&0.4\\
\hline
11	&计算无穷限的反常积分	&	12&	0.6&	0.8\\
\hline
12	&利用微分中值定理证明问题	&	10	&0.8&	0.9\\
\hline

\end{tabular}

试卷总难度:0.55

试卷总区分度:5.58

知识点覆盖率70.5\%

由此看出第一份试卷难度中等,具有一定区分度,知识点覆盖较广;

第二份试卷难度中等偏上,区分度较好,知识点覆盖比较广;

因此综合三项指标,体现了一定的合理性。
\section{数学模型的优缺点评价}

\textbf{针对问题一的三个数学模型}

第一种方法从拟考点的相似度和题目出现的概率出发,分析相似知识点可以提高试卷质量,计算概率可以增加准确性,使得模型更加客观,但得到的数据不够集中。

第二种方法从每个考点的属性本身出发,对七种准则进行了考量,但计算量太大,不好实施。

第三种方法从权重出发。结合了层次分析法的思想,能够得出具体的考点,但是缺点在于判断矩阵时具有主观性。

总体来说第一种方法的科学性最强,第二种方法的计算量较大,第三种方法的主观性最大。

\textbf{针对问题三的最终模型}

这种模型结合了三种方法的优点,具有客观性、准确性,且计算量小,唯一缺点是可能会造成一些知识点的缺失。

\section{参考文献}
[1]李洪福 李振来,试题难度预估方法的探索与实践,生物学通报,第41卷第七期:45,2006

[2]谢志强,题库系统中试卷生成与分析的研究


\end{document}




