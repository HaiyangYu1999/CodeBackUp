\documentclass{ctexart}
\pagestyle{plain}
\usepackage{makecell,rotating,multirow,diagbox,float}
\usepackage{indentfirst}
\usepackage{tabularx}
\usepackage{supertabular}
\usepackage{amsmath}
\date{}
\title{大学生综合素质评价模型建立与探讨}
\begin{document}
\maketitle
\section*{摘要}
本文主要研究了大学生综合素质评价的评分排序问题,基于对45名学生的各项成绩进行分析整合,通过三种方法实现了对大学生综合素质测评的排序求解。本文运用了专家综合分析法,层次分析法,以及Excel软件,对各项指标的影响因素进行了分析,确定了各项指标所占比重,再利用问题本身所具有的数据进行计算得到最终的排名。我们在模型验证过程中发现模型本身对于各项指标的评价因素依然存在不稳定性,对于这些不足我们在模型评价上做了一点建议。
\section*{关键词}
层次分析法\ \ 抽样调查\ \ 专家综合分析法\ \ Excel计算

\newpage
\section{问题提出与重述}
学生综合测评因为包含了诸如思想,健康,品德等不可具体测量因素,简单的定性量化难免会有主观性与局限性的弊端,所以对于学生的综合测评工作需要更加严谨更加有效公平的评测模式。对于题目所给出的信息我们应该找到并分类所提供的评价要素,由于学生综合素质评价的动态性和复杂性,实际执行过程中还会有标准模糊,难以定量化和有效界定等问题,导致评价缺乏科学性,所以如何确立各因素的权值以及如何对要素进行综合分析是问题的关键,在以上问题解决后还应考虑如何对数据进行整理综合以得出最终结果,即确定学生的综合排名。

\section{问题分析}

由题可知,年度综合奖学金等等评定等活动是“激励学生勤奋学习、立志成才,促进德 智体美全面发展,培养合格人才”的奖励办法,其目的是为了以一个公平公正的方法对学生综合能力的排名,我们首先需要建立一套评价体系,确定体系中各因子所占的权重以及各因素对结果的影响程度。在保证符合题干要求的前提下,对于所有数据进行再整理,保证测评的公正性以及科学性。然后通过对数据的计算分析对所有同学的分数进行汇总排名。
\section{建立模型}
\subsection{确立评价因素}
建立完善的评测体系首先需要确定评价要素。归纳每一位大学生在德智体美劳各方面的能力指数,进行多层面的技术评价和总结。
我们首先将\emph{基本素质,课程平均成绩,各创新素质与实践能力}设为第一要素,分别记为$F_{1}$,$F_{2}$,$F_{3}$,在\emph{基本素质}中将\emph{思想政治表现 , 道德品质修养 ,学习态度作风,  组织纪律观念 , 身心健康素质 }设定为第二要素,分别记为$A_{1}$,$A_{2}$,$A_{3}$,$A_{4}$,$A_{5}$,在\emph{各创新素质与实践能力}中将\emph{职业技能, 学科竞赛,科技学术活动,文学艺术作品, 社会实践, 社会工作,文体艺术竞赛}设定为第二要素,记为$B_{1}$,$B_{2}$,$B_{3}$,$B_{4}$,$B_{5}$,$B_{6}$,$B_{7}$,。


\subsection{设立评价指标权值}
对于权值问题,由于直接给出权数指标,难以保证权数的合理性,所以我们通过三种不同的方法,分别求出一套评价标准,再对综合成绩进行计算排名。
\subsection{符号说明}

$a_{i}$---各因素评价指标所得分

$n$---因素个数

$\alpha_{i}$---各因素权重

$Ach$---学生综合评价得分

并定义
$Ach=\sum\limits_{i=1}^{n}\alpha_{i}a_{i}$

\subsection{模型的建立}




\textbf{方法1:抽样调查法}

一般情况下,各高校经过多年对学生综合测评的实施,因此各高校的做法可以较为客观地反映各个因素的重要程度。本方法通过抽样调查的方式搜集了各大高校对各个因素比重的估计,具有较高的参考价值。我们对5所985高校各个因素的比重求平均值,得到一组较有参考价值的比重
\begin{table}[h]\small
\begin{tabular}{|c|c|c|c|c|c|c|c| }
\hline
 一级因素&	二级因素	&1&	2&	3	&4	&5&	平均权重\\
\hline
\multirow{5}{*}{$F_{1}$}&$A_{1}$& 	0.02&	0.02&	0.01&	0.02&	0.02&	0.018\\
\cline{2-8}
&$A_{2}$	&0.02	&0.02&	0.01	&0.02&	0.02&	0.018\\
\cline{2-8}
&$A_{3}$	&0.02&	0.02	&0.01	&0.02	&0.02	&0.018\\
\cline{2-8}
&$A_{4}$	&0.02	&0.02	&0.01&	0.01&	0.02	&0.016\\
\cline{2-8}
&$A_{5}$&0.02	&0.02	&0.01	&0.02	&0.02	&0.018\\
\hline
$F_{2}$&&	0.8	&0.7	&0.85	&0.8&	0.75	&0.78\\

\hline
\multirow{7}{*}{$F_{3}$}&$B_{1}$	&0.01	&0.03	&0.01&	0.01&	0.02	&0.016\\
\cline{2-8}
&$B_{2}$&	0.02&	0.03&	0.02&	0.02	&0.03	&0.024\\
\cline{2-8}
&$B_{3}$&	0.02	&0.03	&0.02&	0.02&	0.02&	0.022\\
\cline{2-8}
&$B_{4}$&	0.02	&0.03	&0.01	&0.02&	0.02&	0.02\\
\cline{2-8}
&$B_{5}$&	0.01	&0.03	&0.01	&0.01&	0.02&	0.016\\
\cline{2-8}
&$B_{6}$&	0.01	&0.02	&0.01&	0.01&	0.02&	0.014\\
\cline{2-8}
&$B_{7}$&	0.01&	0.03	&0.02	&0.02&	0.02	&0.02\\
\hline
\end{tabular}
\end{table}

\newpage
\textbf{方法2:层次分析法}


为了确定$n$个因素在评奖时所占的比重,我们利用层次分析法对$n$个因素的重要程度两两判断,共计$\frac{n(n-1)}{2}$次判断,而不是把每个因素和某个因素比较。这样提供了更多的信息,避免了因为某个因素的判断失误导致的不合理的比重产生。

定义判断矩阵$A=(a_{ij})_{n\times n}$,其中$a_{ij}$表示因素$x_{i}$和$x_{j}$对整体的影响之比,并引入刻画$a_{ij}$的标度1-9.

\begin{table}[h]\small
\begin{tabular}{|c|c| }
\hline
标度&含义\\
\hline
1&表示两个因素相比,具有相同重要性\\
\hline
3&表示两个因素相比,前者比后者稍重要\\
\hline
5&表示两个因素相比,前者比后者明显重要\\
\hline
7&表示两个因素相比,前者比后者强烈重要\\
\hline
9&表示两个因素相比,前者比后者极端重要\\
\hline
2,4,6,8&表示上述相邻判断的中间值\\

\hline

\end{tabular}
\end{table}

接下来为保证主观数据的合理性,
我们需要计算判断矩阵的一致性比例$CR$,在$CR<0.10$ 时
认为主观判断是合理的,然后计算出$A$ 相对于最大特征值$\lambda$的特征向量$\overrightarrow{\xi}$,经归一化之后即是各个因素所占的比重。

考虑到\emph{重要课程成绩}$F_{2}$占的比重相对于\emph{基本素质得分}$F_{1}$和\emph{创新素质与实践能力得分}$F_{3}$中的每一个因素$A_{i}$ 和$B_{i}$ 具有绝对优势,为了减少误差,我们先考虑$F_{1}$,$F_{2}$,$F_{3}$整体所占的比重,然后在从$F_{1}$和$F_{3}$中确定每个因素$A_{i}$和$B_{i}$ 所确定的比重。

首先确定$F_{1}$,$F_{2}$,$F_{3}$的重要程度之比,构造出它们的判断矩阵

$$
A=\left(
\begin{array}{ccc}
1 &5& 2\\
    \frac{1}{4} &1& \frac{1}{2}\\
    \frac{1}{2}& 2& 1\\
\end{array}
\right)
$$

计算出最大特征值$\lambda_{max}=3.05$

对应的特征向量$$\overrightarrow{\xi}=(0.8883,0.4108,0.2054)$$
然后计算一致性比例$CR=\frac{a-3}{0.58}=0.086<0.10$,所以我们可以认为特征向量各个分量的比例为$F_{1}$,$F_{2}$,$F_{3}$的比重,大致为

$$\begin{tabular}{|c|c| }
\hline
因素&比重\\
\hline
$F_{1}$ & 13\% \\
\hline
$F_{2} $& 60\% \\
\hline
$F_{3}$  &27\% \\
\hline
\end{tabular}
$$
接下来需要计算$F_{1}$和$F_{3}$中每个因素$A_{i}$和$B_{i}$所占的比重,考虑到$F_{1}$在总体中的比重非常低并且$F_{1}$中的各因素$A_{i}$ 的重要程度十分相近,我们近似认为$F_{1}$中的每个元素占的比重相同。
对于$F_{3}$中的$B_{i}$,使用相同的方法构造判断矩阵
$$B=\left(
\begin{array}{ccccccc}
1&		3&2&	 \frac{1}{2}	&	 \frac{1}{2}	&	 \frac{1}{2}&		 \frac{1}{2}\\
 \frac{1}{2}	&	1	&	 \frac{1}{2}&		\frac{1}{3}	&	 \frac{1}{3}	&	 \frac{1}{3}&		 \frac{1}{3}\\
\frac{1}{2}&		2&		1	 &	\frac{1}{2} &	\frac{1}{2}	 &	\frac{1}{2}&		\frac{1}{2}\\
2&		3&		2	&	1	&	1	&	1&		1\\
2&		3	&	2	&	1&		1	& \frac{1}{2}	&1\\
1	&	3	&	2	&		1	&	1&		1&1\\
2	&	3	&	2	&		1&		1&		1	&	1\\
\end{array}
\right)
$$


计算得最大特征值$\lambda=6.9860$

对应的特征向量
$$
\overrightarrow{p}=\left(
\begin{array}{c}
0.3015\\
0.1459\\
0.2066\\
0.4754\\
0.4445\\
0.4323\\
0.4754\\
\end{array}
\right)
$$




一致性比例CR=0.024<0.10,有较好的近似性。

故$B_{1}$到$B_{7}$的比重可近似为

\begin{tabular}{|c|c|c|c|c|c|c|c| }
\hline
因素&$B_{1}$&$B_{2}$&$B_{3}$&$B_{4}$&$B_{5}$&$B_{6}$&$B_{7}$\\
\hline
比重&18\%&  19\%&  19\% & 6\%&  8\%&  18\% & 12\% \\
\hline

\end{tabular}

\newpage
综合$F_{1}$,$F_{2}$,$F_{3}$的比重和$A_{i}$,$B_{i}$的比重,得到每个因素所占的比重


\begin{table}[htbp]
\begin{tabular}{|c|c|c|c|c|c|c|c|}
\hline
\multirow{2}{*}{因素}&\multicolumn{5}{|c|}{$F_{1}$(13\%)}&\multicolumn{2}{|c|}{$F_{2}$(60\%)}\\
\cline{2-8}
&$A_{1}$	  &  $A_{2}$	 & $A_{3}$&$A_{4}$  & $A_{5}$& 	\multicolumn{2}{|c|}{$F_{2}$}\\
\hline
比重&2.60\%&	2.60\%	&2.60\%&	2.60\%	&2.60\%&\multicolumn{2}{|c|}{60\%}\\
\hline
\multirow{2}{*}{因素}&\multicolumn{7}{|c|}{$F_{3}$(27\%)}\\
\cline{2-8}
&$B_{1}$&$B_{2}$&$B_{3}$&$B_{4}$&$B_{5}$&$B_{6}$&$B_{7}$\\
\hline
比重&4.86\%&5.13\%&	5.13\%&	1.62\%&	2.16\%&	4.86\%&	3.24\%\\
\hline

\end{tabular}
\end{table}

\textbf{方法3:}
一般情况下,教师和辅导员会对学生的学习情况有着准确的了解,因此他们的意见可以较为客观的反映各个因素的重要程度。本方法通过抽样调查的方式调查了不同院系的教师和辅导员对各个因素比重的估计,具有较高的参考价值。我们将10位教师和辅导员给出的每个因素的比重求平均值,得到一组较有参考价值的比重,如下表所示。

\begin{table}[h]\scriptsize
\begin{tabular}{|c|c|c|c|c|c|c|c|c|c|c|c|}
\hline
\diagbox{因素}{教师}&1&2&3&4&5&6&7&8&9&10&平均比重\\
\hline
A1 & 	0.02&	0.02&	0.03&	0.01&	0.02&	0.02&	0.02&	0.03	&0.01	&0.02&	0.02\\
\hline
A2& 	0.02	&0.02&	0.02&	0.03&	0.01&	0.02&	0.02&	0.02	&0.03&	0.01&	0.02\\
\hline
A3&	0.02&	0.02	&0.02	&0.03&	0.01&	0.02&	0.02&	0.02&	0.03&	0.01&	0.02\\
\hline
A4& 	0.02&	0.02&	0.02&	0.03	&0.01	&0.02&	0.02&	0.02&	0.03	&0.01&	0.02\\
\hline
A5& 	0.02&	0.02&	0.02&	0.03&	0.01&	0.02&	0.02&	0.02&	0.03&	0.01&	0.02\\
\hline
F2&	0.76	&0.69	&0.66	&0.64&	0.85	&0.69	&0.73	&0.7	&0.71&	0.6&	0.703\\
\hline
B1 &	0.02&	0.03&	0.03&	0.03&	0.02&	0.03&	0.02&	0.02&	0.02	&0.05&	0.027\\
\hline
B2 &	0.02&	0.03&	0.04&	0.03&	0.02&	0.03&	0.03&0.05&	0.02	&0.05&	0.032\\
\hline
B3&	0.02&	0.03	&0.03&0.03&	0.02&	0.03&	0.04&	0.05&	0.02&	0.05&	0.032\\
\hline
B4 &	0.02&	0.03&	0.03	&0.03&	0.01&	0.03&	0.02&	0.02&	0.02&	0.05&	0.026\\
\hline
B5	&0.02&	0.03	&0.04	&0.03	&0.01	&0.03	&0.02	&0.02	&0.02	&0.05&	0.027\\
\hline
B6 &	0.02	&0.03&	0.04&	0.03&	0.01&	0.03&	0.02&	0.02&	0.02&	0.05	&0.027\\
\hline
B7&	0.02&	0.03	&0.03	&0.03	&0.01	&0.03	&0.02	&0.02	&0.02&	0.05&	0.026\\
\hline
\end{tabular}
\end{table}



\subsection{模型求解}

由三种方法所得结果可知各因素对于综合评价重要性所占比例,分别将对应数据代入Excel运用sumproduct函数计算所有数据的权值(这里使用Excel可以更加快速方便的求出多组数据的结果)由于单一数据的不稳定性,我们可以通过对于多组数据求平均数来提高准确率,求出平均数后得到了所有学生的Ach综合测评成绩,再运用Rank函数对所有成员进行综合排名得到结果如下表所示。

\newpage
\begin{center}
\begin{table}[H]\scriptsize
\begin{tabular}{|c|c|c|c|c|c|c|}
\hline
学生 & \multicolumn{2}{|c|}{方法1}& \multicolumn{2}{|c|}{方法2} & \multicolumn{2}{|c|}{方法3} \\
\cline{2-7}
排名&	学生编号&	得分	&学生编号	&得分	&学生编号&	得分\\
\hline
  1 	& 2 	& 82.702 	& 2 	& 78.431 	& 2 	& 80.024\\\hline
 2 	& 3 	& 80.474 	& 1 	& 76.247 	& 1 	& 78.798\\\hline
 3 	& 1 	& 79.796 	& 6 	& 74.073 	& 3 	& 77.027\\\hline
 4 	& 5 	& 79.544 	& 5 	& 72.760 	& 5 	& 76.154\\\hline
 5 	& 4 	& 75.938 	& 3 	& 72.492 	& 4 	& 74.042\\\hline
 6 	& 6 	& 74.926 	& 4 	& 72.012 	& 6 	& 73.996\\\hline
 7 	& 7 	& 14.364 	& 7 	& 25.843 	& 7 	& 18.896\\\hline
 8 	& 12 	& 13.286 	& 12 	& 25.119 	& 12 	& 17.118\\\hline
 9 	& 21 	& 12.920 	& 13 	& 23.628 	& 21 	& 16.565\\\hline
 10 	& 15 	& 12.828 	& 25 	& 23.414 	& 15 	& 16.433\\\hline
 11 	& 25 	& 12.748 	& 16 	& 22.761 	& 25 	& 16.420\\\hline
 12 	& 16 	& 12.668 	& 15 	& 22.289 	& 13 	& 16.364\\\hline
 13 	& 13 	& 12.540 	& 21 	& 21.960 	& 16 	& 16.274\\\hline
 14 	& 8 	& 12.258 	& 31 	& 20.995 	& 8 	& 15.496\\\hline
 15 	& 31 	& 12.048 	& 29 	& 20.346 	& 14 	& 14.995\\\hline
 16 	& 34 	& 11.904 	& 19 	& 20.287 	& 31 	& 14.932\\\hline
 17 	& 14 	& 11.890 	& 34 	& 20.153 	& 34 	& 14.787\\\hline
 18 	& 37 	& 11.792 	& 38 	& 19.876 	& 29 	& 14.762\\\hline
 19 	& 29 	& 11.734 	& 40 	& 19.836 	& 37 	& 14.725\\\hline
 20 	& 9 	& 11.710 	& 30 	& 19.678 	& 27 	& 14.655\\\hline
 21 	& 35 	& 11.682 	& 8 	& 19.678 	& 40 	& 14.634\\\hline
 22 	& 19 	& 11.646 	& 23 	& 19.634 	& 19 	& 14.626\\\hline
 23 	& 27 	& 11.644 	& 9 	& 19.591 	& 9 	& 14.588\\\hline
 24 	& 45 	& 11.534 	& 26 	& 19.460 	& 35 	& 14.563\\\hline
 25 	& 33 	& 11.512 	& 14 	& 19.451 	& 30 	& 14.516\\\hline
 26 	& 30 	& 11.484 	& 35 	& 19.159 	& 33 	& 14.466\\\hline
 27 	& 36 	& 11.446 	& 37 	& 19.095 	& 45 	& 14.409\\\hline
 28 	& 38 	& 11.398 	& 24 	& 19.093 	& 24 	& 14.392\\\hline
 29 	& 10 	& 11.360 	& 45 	& 18.706 	& 36 	& 14.263\\\hline
 30 	& 17 	& 11.356 	& 43 	& 18.705 	& 38 	& 14.224\\\hline
 31 	& 24 	& 11.334 	& 27 	& 18.668 	& 10 	& 14.212\\\hline
 32 	& 40 	& 11.292 	& 33 	& 18.663 	& 17 	& 14.099\\\hline
 33 	& 26 	& 11.088 	& 36 	& 18.644 	& 26 	& 13.923\\\hline
 34 	& 23 	& 11.078 	& 18 	& 18.468 	& 23 	& 13.909\\\hline
 35 	& 43 	& 10.964 	& 22 	& 18.387 	& 11 	& 13.752\\\hline
 36 	& 11 	& 10.948 	& 10 	& 18.247 	& 43 	& 13.552\\\hline
 37 	& 18 	& 10.838 	& 28 	& 17.998 	& 42 	& 13.469\\\hline
 38 	& 42 	& 10.814 	& 11 	& 17.866 	& 18 	& 13.446\\\hline
 39 	& 22 	& 10.774 	& 20 	& 17.748 	& 22 	& 13.389\\\hline
 40 	& 28 	& 10.610 	& 32 	& 17.624 	& 28 	& 13.134\\\hline
 41 	& 32 	& 10.322 	& 41 	& 17.575 	& 44 	& 12.915\\\hline
 42 	& 44 	& 10.270 	& 17 	& 17.560 	& 39 	& 12.808\\\hline
 43 	& 20 	& 10.236 	& 42 	& 17.425 	& 32 	& 12.694\\\hline
 44 	& 39 	& 10.042 	& 44 	& 16.939 	& 20 	& 12.686\\\hline
 45 	& 41 	& 9.914 	& 39 	& 16.614 	& 41 	& 12.402\\\hline
\end{tabular}
\end{table}
\end{center}

\section{模型评价与推广}
此模型通过对学生多项素质的分析考察,以一个量化的方式对每一位同学的综合成绩进行了排名,由于大部分工作都可以通过计算机进行计算,所以保证了测评的准确性,也提高了效率,在具体实施方面具有一定的可操作性与实用性,多种方式的求解也使得结果更有说服力,更具科学性。

但由于综合测试中类似于道德,思想等因素的不稳定性以及量化所带来了的模糊感,使得测试结果存在较大的误差风险,此评价的合理性很大程度上依赖于专家经验和引导的倾向性,这需要不断深入学生的成长和规律,通过更加细致的多层次分析,再结合更多更加精细的模糊因素评价以及更多更细致的因素做出一套更加科学的综合测评体系,再充分利用综合素质评价的积极作用,使其发挥其应有的水平,达到勉励学生促进积极进步的作用。

\section{参考文献}
[1]陈华友,周礼刚, 刘金陪,  数学模型与数学建模\ 北京出版社

[2]董卓宁\ 张江\ 张弛\ 大学生综合素质体系构建与实施方法研究\ 北京航空航天大学

[3]现代统计学\ 数学建模各种分析方法

[4]姜启源,谢金星,叶俊,数学模型(第三版),北京市高等教育出版社



\end{document}




