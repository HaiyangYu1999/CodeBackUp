\documentclass{ctexart}
\usepackage{amsmath,amssymb}
\DeclareSymbolFont{EulerExtension}{U}{euex}{m}{n}
\DeclareMathSymbol{\euintop}{\mathop} {EulerExtension}{"52}
\DeclareMathSymbol{\euointop}{\mathop} {EulerExtension}{"48}
\let\intop\euintop
\let\ointop\euointop
\pagestyle{plain}

\begin{document}

有这样一类题,是求重积分的极限问题,并且重积分的边界趋于无穷.
这样的题有一些简化积分区域的小技巧
\newline
\newline
\textbf{例:}

设矩阵$A=(a_{ij})_{n\times n}$正定矩阵,计算极限
$$\lim_{R\rightarrow \infty}\int\int\cdots\int_{x_{1}^{2}+x_{2}^{2}+\cdots +x_{n}^{2}\leq R^{2}}\mathrm{e}^{-\sum\limits_{1\leq i,j\leq n}a_{ij}x_{i}x_{j}}\mathrm{d}x_{1}\mathrm{d}x_{2}\cdots\mathrm{d}x_{n}$$
\newline
一看到正定矩阵,就要想到实矩阵的对角化并且特征值全部大于0,并且变换矩阵是正交矩阵,那自然少不了正交换元.设$$A=P^{-1}\mathrm{diag}(\lambda_{1},\lambda_{2},\cdots,\lambda_{n})P$$那么$P$就是正交矩阵了,所以作换元
\begin{equation}\nonumber
{
\left( \begin{array}{c}
x_{1}\\
x_{2}\\
\vdots\\
x_{n}\\
\end{array}
\right)=P\left( \begin{array}{c}
y_{1}\\
y_{2}\\
\vdots\\
y_{n}\\
\end{array}
\right)}
\end{equation}


接下来...你应该懂得(手动滑稽)
,就变为了
$$\lim_{R\rightarrow \infty}\int\int\cdots\int_{y_{1}^{2}+y_{2}^{2}+\cdots +y_{n}^{2}\leq R^{2}}\mathrm{e}^{-\lambda_{1}y_{1}^{2}-\lambda_{2}y_{2}^{2}\cdots-\lambda_{n}y_{n}^{2}}\mathrm{d}y_{1}\mathrm{d}y_{2}\cdots\mathrm{d}y_{n}$$

这个积分仍然不方便算出,原因就是积分区域.假如积分区域为$\mathbb{R}^{n}$就很容易积出了,但是,通过类比
二维圆$$\{(x,y):x^{2}+y^{2}\leq R^{2}\}\subseteq\{(x,y):x\leq |R|,y\leq |R|\}\subseteq\{(x,y):x^{2}+y^{2}\leq 2R^{2}\}$$
即一个正方形总在一个它的内接圆和一个外接圆之间.

对于三维球$$\{x^{2}+y^{2}+z^{2}\leq R^{3}\}\subseteq\{x\leq |R|,y\leq |R|,z\leq |R|\}\subseteq\{x^{2}+y^{2}+z^{2}\leq 3R^{3}\}$$
即一个正方体总在一个它的内接球和一个外接球之间.

对于n维球,也有$$\{x_{1}^{2}+x_{2}^{2}+x_{3}^{2}\leq R^{n}\}\subseteq\{x_{1}\leq |R|,x_{2}\leq |R|,\cdots,x_{n}\leq |R|\}\subseteq\{x_{1}^{2}+x_{2}^{2}+\cdots+x_{n}^{2}\leq nR^{n}\}$$

又因为被积函数在任何区域都为正,所以$$\int_{x_{1}^{2}+x_{2}^{2}+x_{3}^{2}\leq R^{n}}f\mathrm{d}u\leq\int_{x_{1}\leq |R|,x_{2}\leq |R|,\cdots,x_{n}\leq |R|}f\mathrm{d}u\leq\int_{x_{1}^{2}+x_{2}^{2}+x_{3}^{2}\leq nR^{n}}f\mathrm{d}u$$

当$R\rightarrow +\infty$时,假如极限$$\lim_{R\rightarrow +\infty}\int_{x_{1}^{2}+x_{2}^{2}+x_{3}^{2}\leq R^{n}}f\mathrm{d}u$$存在,由夹挤定理知$$\lim_{R\rightarrow +\infty}\int_{x_{1}\leq |R|,x_{2}\leq |R|,\cdots,x_{n}\leq |R|}f\mathrm{d}u=\lim_{R\rightarrow +\infty}\int_{x_{1}^{2}+x_{2}^{2}+x_{3}^{2}\leq R^{n}}f\mathrm{d}u$$

所以就把积分区域简化为$[-\infty,+\infty]^{n}$

然后只需计算$$\int^{+\infty}_{-\infty}\int^{+\infty}_{-\infty}\cdots\int^{+\infty}_{-\infty}
\mathrm{e}^{-\lambda_{1}y_{1}^{2}-\lambda_{2}y_{2}^{2}\cdots-\lambda_{n}y_{n}^{2}}\mathrm{d}y_{1}\mathrm{d}y_{2}\cdots\mathrm{d}y_{n}$$

即$$\prod_{i=1}^{n}\int^{+\infty}_{-\infty}\mathrm{e}^{-\lambda_{i}y_{i}^{2}}\mathrm{d}y_{i}$$



虽然n重积分基本上不会考到,但是利用极限简化积分区域这种方法还是有用的.
可以在2,,3维半径趋于无穷的球或者椭球在球坐标变换不方便时试用
也可以是在不太规则的距离远点距离趋于无穷的几何形体上使用.
\end{document}
