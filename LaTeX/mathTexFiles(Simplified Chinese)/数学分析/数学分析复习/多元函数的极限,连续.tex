\documentclass{ctexart}
\usepackage{amsmath,amssymb}
\DeclareSymbolFont{EulerExtension}{U}{euex}{m}{n}
\DeclareMathSymbol{\euintop}{\mathop} {EulerExtension}{"52}
\DeclareMathSymbol{\euointop}{\mathop} {EulerExtension}{"48}
\let\intop\euintop
\let\ointop\euointop
\pagestyle{plain}
\begin{document}
\subsection*{1.求多元函数的极限}
\begin{flalign}
(1)&\lim_{\substack{x\rightarrow 0\\y\rightarrow 0}}(x^{2}+y^{2})^{x^{2}y^{2}}&\\
(2)&\lim_{\substack{x\rightarrow 0\\y\rightarrow 0}}\frac{\mathrm{ln}(x^{2}+\mathrm{e}^{y^{2}})}{x^{2}+y^{2}}&\\
(3)&\lim_{\substack{x\rightarrow 0\\y\rightarrow 0}}\frac{1-\mathrm{cos}(x^{2}+y^{2})}{(x^{2}+y^{2})x^{2}y^{2}}&\\
(4)&\lim_{\substack{x\rightarrow 0\\y\rightarrow 0}}\frac{\sqrt{1+xy}-1}{xy}&\\
(5)&\lim_{x\rightarrow +\infty}\lim_{y\rightarrow+\infty}\mathrm{sin}\ \frac{\pi x}{2x+y}\ \ \ \lim_{y\rightarrow +\infty}\lim_{x\rightarrow+\infty}\mathrm{sin}\ \frac{\pi x}{2x+y}
\end{flalign}
\subsection*{2.讨论下列函数是否在(0,0)处连续}
\begin{flalign}
(1)&f(x,y)=\frac{x^{6}y^{8}}{(x^{2}+y^{4})^{5}}&\\
(2)&f(x,y)=\frac{x^{3}y^{3}}{x^{4}+y^{8}}&\\
(3)&f(x,y)=\frac{x^{2}y}{x^{2}+y^{2}}&\\
(4)&f(x,y)=\frac{xy}{x^{2}+y^{2}}&\\
(5)&f(x,y)=\frac{xy}{x+y}&\\
(6)&f(x,y)=\frac{xy}{\sqrt{1+x+y}-1}&
\end{flalign}
\subsection*{3.说明下列函数在(0,0)处的二重极限和累次极限是否存在}
\begin{flalign}
(1)&f(x,y)=(x+y)\mathrm{sin}\ \frac{1}{x}\ \mathrm{sin}\ \frac{1}{y}&\\
(2)&f(x,y)=\frac{x^{2}y^{2}}{x^{2}y^{2}+(x-y)^{2}}&\\
(3)&f(x,y)=\frac{x^{2}(1+x^{2})-y^{2}(1+y^{2})}{x^{2}+y^{2}}&
\end{flalign}
\subsection*{4.证明下列函数不一致连续}
\begin{flalign}
&f(x,y)=\mathrm{sin}\ \frac{1}{x+y}\ \ \ \ \ \ (x,y)\in [0,1]^{2}\setminus(0,0)&
\end{flalign}
\subsection*{5.方向导数,偏导数,可微}
\noindent(1)设函数$f(x,y)=xy.$计算函数$f$在点$(1,1)$处沿方向$\mathbf{u}=(1,1)$的方向导数.

\noindent(2)设函数$f(x,y)=(x-1)^{2}-y^{2}.$计算函数$f$在点$(0,1)$处沿方向$\mathbf{u}=(3/5,-4/5)$的方向导数.

\noindent ps:我觉得初等多元函数偏导数的计算你应该会了(比如:$z=x^{y}$,求$\frac{\partial z}{\partial x}$和$\frac{\partial z}{\partial y}$),我就不找题了.

\noindent (3)设
\begin{eqnarray}
f(x,y)=
\begin{cases}
\frac{xy}{x^{2}+y^{2}},&x^{2}+y^{2}>0,\\
0,&x^{2}+y^{2}=0.
\end{cases}
\end{eqnarray}
验证:$f$在$(0,0)$ 处不连续,因而$f$在$(0,0)$处不可微.并验证$\frac{\partial f}{\partial x}(x,y)$,$\frac{\partial f}{\partial y}(x,y)$在$(0,0)$ 处不连续.

\noindent (4)证明:函数$f(x,y)=\sqrt{|xy|}$在$(0,0)$处不可微.

\noindent (5)验证函数
\begin{eqnarray}
f(x,y)=
\begin{cases}
\frac{xy}{\sqrt{x^{2}+y^{2}}},&x^{2}+y^{2}>0,\\
0,&x^{2}+y^{2}=0.
\end{cases}
\end{eqnarray}
在$(0,0)$连续且偏导数存在,但在$(0,0)$处不可微.

\noindent (6)设
\begin{eqnarray}
f(x,y)=
\begin{cases}
(x^{2}+y^{2})\mathrm{sin}\ \frac{1}{x^{2}+y^{2}},&x^{2}+y^{2}>0,\\
0,&x^{2}+y^{2}=0.
\end{cases}
\end{eqnarray}
验证:$\frac{\partial f}{\partial x}$,$\frac{\partial f}{\partial y}$在$(0,0)$处不连续,但$f$在$(0,0)$处可微.
\newpage

\textbf{证明一个函数是否可微是重点}

\textbf{偏导数连续是最强的条件,可以推出可微,不可微也可以推出偏导数不连续,但是可微不能推出偏导数连续,偏导数不连续也不能推出不可微(如(6))。}

\textbf{可微可以同时推出函数连续和各方向导数(偏导数)存在,但是函数连续不能推出可微,各方向导数存在(偏导数存在)也不能推出可微。
所以偏导数不存在或方向导数不存在可以推出不可微,不连续可以推出不可微}


\textbf{各方向导数存在或偏导数存在和函数连续没有必然的联系},例如: 当$y=x^{2}$且$x\neq 0$时 ,f(x,y)=1,其他情况时,f(x,y)=0.对于这样的f 在(0,0) 处偏导数和各方向导数存在但f 不连续,函数连续也不能推出偏导数或方向导数存在。

\textbf{综上}

\textbf{证明一个函数不可微常用的有三种方法,一是用不连续推出不可微,二是利用偏导数不存在推出不可微,假如f既连续,偏导数又存在就只能在求出偏导数之后利用定义来证明不可微},即证明
$$f(x+\Delta x,y+\Delta y)-\frac{\partial f}{\partial x}(x,y)\cdot \Delta x-\frac{\partial f}{\partial y}(x,y)\cdot \Delta y-f(x,y)$$不是$\sqrt{x^{2}+y^{2}}$的高阶无穷小,即$$\lim_{\rho\rightarrow 0}\frac{f(x+\Delta x,y+\Delta y)-\frac{\partial f}{\partial x}(x,y)\cdot \Delta x-\frac{\partial f}{\partial y}(x,y)\cdot \Delta y-f(x,y)}{\sqrt{x^{2}+y^{2}}}$$不为0

\textbf{证明一个函数可微只有用定义(虽然偏导数连续也能推出可微但是我做过的题里都没有这么用的)}
先计算出偏导数$\frac{\partial f}{\partial x}$和$\frac{\partial f}{\partial y}$再证明$$f(x+\Delta x,y+\Delta y)-\frac{\partial f}{\partial x}(x,y)\cdot \Delta x-\frac{\partial f}{\partial y}(x,y)\cdot \Delta y-f(x,y)$$是$\sqrt{x^{2}+y^{2}}$的高阶无穷小,即$$\lim_{\rho\rightarrow 0}\frac{f(x+\Delta x,y+\Delta y)-\frac{\partial f}{\partial x}(x,y)\cdot \Delta x-\frac{\partial f}{\partial y}(x,y)\cdot \Delta y-f(x,y)}{\sqrt{x^{2}+y^{2}}}=0$$




\textbf{答案}

1.(1) 1

1.(2) 1

1.(3) $+\infty$

1.(4) 1/2

1.(5) 1 \ \ 0

2.除了(3)外,都不连续

3.(1)极限存在,累次极限均不存在

3.(2)极限不存在,累次极限存在

3.(3)极限不存在,累次极限存在

4.考虑点列$(x_{n},y_{n})=(\frac{1}{\pi+4n\pi},\frac{1}{\pi+4n\pi})$,$(x'_{n},y'_{n})=(\frac{1}{4n\pi},\frac{1}{4n\pi})$
$n\rightarrow +\infty$时,虽然两点的距离趋于0,但$f(x_{n},y_{n})-f(x'_{n},y'_{n})=1$
\end{document}
