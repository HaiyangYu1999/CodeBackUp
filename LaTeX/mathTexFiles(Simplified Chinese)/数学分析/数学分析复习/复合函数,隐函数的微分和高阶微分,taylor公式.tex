\documentclass{ctexart}
\usepackage{amsmath,amssymb}
\DeclareSymbolFont{EulerExtension}{U}{euex}{m}{n}
\DeclareMathSymbol{\euintop}{\mathop} {EulerExtension}{"52}
\DeclareMathSymbol{\euointop}{\mathop} {EulerExtension}{"48}
\let\intop\euintop
\let\ointop\euointop
\pagestyle{plain}
\begin{document}

\subsection*{1.复合求导}


\noindent (1)设$u(x,y)$有连续的一阶偏导数,又设$x=r\ \mathrm{cos}\ \theta$,$y=r\ \mathrm{sin}\ \theta$,证明:
$$(\frac{\partial u}{\partial x})^{2}+(\frac{\partial u}{\partial y})^{2}=(\frac{\partial u}{\partial r})^{2}+\frac{1}{r^{2}}(\frac{\partial u}{\partial \theta})^{2}$$

\noindent (2)设$w=f(x,y,z),x=u+v,y=u-v,z=uv$,求$\frac{\partial w}{\partial u}$和$\frac{\partial w}{\partial v}$

\noindent (3)设$w=f(x,u,v),u=g(y,z),v=h(x,y)$,求$\frac{\partial w}{\partial x}$,$\frac{\partial w}{\partial y}$,$\frac{\partial w}{\partial z}$

\subsection*{2.隐函数求导}

\noindent (1)设$z=z(x,y)$是由方程$$f(x+\frac{z}{y},y+\frac{z}{x})=0$$所确定的隐函数.计算$\frac{\partial z}{\partial x}$,$\frac{\partial z}{\partial y}$

\noindent (2)设$z=z(x,y)$是由方程$$\mathrm{e}^{z}-xyz=0$$所确定的隐函数.计算$\frac{\partial z}{\partial x}$,$\frac{\partial z}{\partial y}$

\noindent (3)设$F(x,y,z)=0$.求证:$\frac{\partial x}{\partial y}\frac{\partial y}{\partial z}\frac{\partial z}{\partial x}=-1$

\noindent (4)设$x=x(u,v),y=y(u,v)$是由方程组$$xu-yv=0,yu+xv=1$$所确定的隐函数,求$\frac{\partial x}{\partial u}$,$\frac{\partial y}{\partial u}$


\subsection*{3.高阶微分}



\noindent (1)设$u=f(r,r\mathrm{cos}\ \theta)$有二阶连续偏导数,求$\frac{\partial u}{\partial r},\frac{\partial u}{\partial \theta},\frac{\partial^{2} u}{\partial r\partial \theta}.$(这个题极其容易做错)

\noindent (2)设方程$x^{2}+y^{2}+z^{2}=4z$确定$z$为$x,y$的函数,求$\frac{\partial^{2} z}{\partial x^{2}},\frac{\partial^{2} z}{\partial x\partial y}$

\subsection*{3.Taylor公式}

\noindent 证明:当$|x|$和$|y|$充分小时,有近似式$$\frac{\mathrm{cos}\ x}{\mathrm{cos}\ y}\approx 1-\frac{1}{2}(x^{2}-y^{2})$$


\textbf{答案}

1.(2)$\frac{\partial w}{\partial u}=\frac{\partial f}{\partial x}+\frac{\partial f}{\partial y}+v\frac{\partial f}{\partial z}$

$\frac{\partial w}{\partial v}=\frac{\partial f}{\partial x}-\frac{\partial f}{\partial y}+u\frac{\partial f}{\partial z}$

1.(3)$\frac{\partial w}{\partial x}=\frac{\partial f}{\partial x}+\frac{\partial f}{\partial v}\frac{\partial h}{\partial x}$

$\frac{\partial w}{\partial y}=\frac{\partial f}{\partial u}\frac{\partial g}{\partial y}+\frac{\partial f}{\partial v}\frac{\partial h}{\partial y}$

$\frac{\partial w}{\partial z}=\frac{\partial f}{\partial u}\frac{\partial g}{\partial z}$

2.(1)$$\frac{\partial z}{\partial x}=\frac{\frac{z}{x^{2}}f'_{2}-f'_{1}}{\frac{f'_{1}}{y}+\frac{f'_{2}}{x}}$$

$$\frac{\partial z}{\partial y}=\frac{\frac{z}{y^{2}}f'_{1}-f'_{2}}{\frac{f'_{1}}{y}+\frac{f'_{2}}{x}}$$

2.(2)$\frac{\partial z}{\partial x}=\frac{yz}{\mathrm{e}^{z}-xy},\frac{\partial z}{\partial y}=\frac{xz}{\mathrm{e}^{z}-xy}$

2.(4)$\frac{\partial x}{\partial u}=-\frac{ux+vy}{u^{2}+v^{2}},\frac{\partial y}{\partial u}=\frac{vx-uy}{u^{2}+v^{2}}$

3.(1)$\frac{\partial u}{\partial r}=f'_{1}+f'_{2}\mathrm{cos}\ \theta$

$\frac{\partial u}{\partial \theta}=-rf'_{2}\mathrm{sin}\ \theta$

$\frac{\partial^{2} u}{\partial r\partial \theta}=-rf''_{12}\mathrm{sin}\ \theta-f'_{2}\mathrm{sin}\ \theta-rf''_{22}\mathrm{sin}\ \theta\mathrm{cos}\ \theta$

3.(2)$\frac{\partial^{2} z}{\partial x^{2}}=\frac{(2-z)^{2}+x^{2}}{(2-z)^{3}}$

$\frac{\partial^{2} z}{\partial x\partial y}=\frac{xy}{(2-z)^{3}}$

\end{document}