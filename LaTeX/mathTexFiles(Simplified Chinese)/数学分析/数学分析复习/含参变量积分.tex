\documentclass{ctexart}
\usepackage{amsmath,amssymb,bm}
\DeclareSymbolFont{EulerExtension}{U}{euex}{m}{n}
\DeclareMathSymbol{\euintop}{\mathop} {EulerExtension}{"52}
\DeclareMathSymbol{\euointop}{\mathop} {EulerExtension}{"48}
\let\intop\euintop
\let\ointop\euointop
\pagestyle{plain}
\begin{document}
\subsection*{1.求极限问题}
这种题的一般形式就是$$\lim_{a \rightarrow \alpha}\int_{a}^{b}f(x,a)\mathrm{d}x$$
这种题真的很简单,
做法就是把极限带入到被积函数中,被积函数求极限后会变得灰常简单.

但要知道这样做的根据.(假如题目要求说明这样做的理由,需要有恰当的解释,下面的定理是这样做的理论依据)(定理也要看看嘛,说不定定理证明就考这一个呢)

\textbf{定理1(极限和积分换序):}如果函数$f$在闭矩形$I=[a,b]\times[\alpha,\beta]$上连续,那么$$\varphi(u)=\int_{a}^{b}f(x,u)\mathrm{d}x$$是区间$[\alpha,\beta]$上的连续函数.

\textbf{证明:}任取$u_{0}\in[\alpha,\beta]$,我们证明$\varphi$在$u_{0}$处连续.从
\begin{equation}
\varphi(u)-\varphi(u_{0})=\int_{a}^{b}(f(x,u)-f(x,u_{0}))\mathrm{d}x
\end{equation}
得到
$$|\varphi(u)-\varphi(u_{0})|\leq \int_{a}^{b}|f(x,u)-f(x,u_{0})|\mathrm{d}x$$
由于$f$在闭矩形上连续,所以必定一致连续,因而对任意的$\varepsilon>0$,存在$\delta>0$,对闭矩形$I$中任意的两点$(x_{1},u_{1}),(x_{2},u_{2})$,只要它们的距离小于$\delta$,就有$$|f(x_{1},u_{1})-f(x_{2},u_{2})|\leq \varepsilon$$
由于点$(x,u)$和点$(x,u_{0})$的距离等于$|u-u_{0}|$,所以当$|u-u_{0}|<\delta$时,$$|f(x,u)-f(x,u_{0})|\leq \varepsilon$$对任意的$x\in[a,b]$成立,于是由$(1)$,即得
$$|\varphi(u)-\varphi(u_{0})|<\varepsilon(b-a)$$
这就证明了$\varphi$在$u_{0}$处连续,由于$u_{0}$是$[\alpha,\beta]$中的任意一点,所以$\varphi$在$[\alpha,\beta]$上连续.

注意,$\varphi$在$u_{0}$连续意味着\begin{equation}\lim_{u\rightarrow u_{0}}\varphi(u)=\varphi(u_{0})\end{equation}
而$$\varphi(u_{0})=\int_{a}^{b}f(x,u_{0})\mathrm{d}x=\int_{a}^{b}\lim_{u\rightarrow u_{0}}f(x,u)\mathrm{d}x$$
这样,式(2)可以写为$$\lim_{u\rightarrow u_{0}}\int_{a}^{b}f(x,u)\mathrm{d}x=\int_{a}^{b}\lim_{u\rightarrow u_{0}}f(x,u)\mathrm{d}x$$
这就是说,$f$的连续性可以保证积分和极限交换顺序.
\newline
\newline
\indent\textbf{练习:}

计算下列极限:

(1)$\lim\limits_{x\rightarrow 0}\int_{-1}^{1}\sqrt{x^{2}+y^{2}}\mathrm{d}y$

(2)$\lim\limits_{t\rightarrow 0}\int_{0}^{2}x^{2}\mathrm{cos}\ tx\mathrm{d}x$

(3)$\lim\limits_{\alpha\rightarrow 0}\int_{\alpha}^{\alpha+\frac{\pi}{2}}\frac{\mathrm{sin}^{2}x}{4+\alpha x^{2}}\mathrm{d}x$

\subsection*{2.积分号下求导问题}

这种题的一般形式就是$$F(u)=\int_{p(u)}^{q(u)}f(x,u)\mathrm{d}x$$
求$F'(u)$

这种题也算比较简单,套公式就可以了,注意不要算错

这样做的根据是:

假如$f$和$\frac{\partial f}{\partial u}$在闭矩形$I=[a,b]\times[\alpha,\beta]$上连续,并且当$u$有界时,$p(u),q(u)$都有界,那么$F$在$[\alpha,\beta]$上可微,并且
$$F'(u)=\int_{p(u)}^{q(u)}\frac{\partial f(x,u)}{\partial u}\mathrm{d}x+f(q(u),u)q'(u)-f(p(u),u)p'(u)$$
(注意:公式中积分号下限的那个函数求导前边的是负号)(看下这个定理的证明,有可能会考)

\textbf{练习:}

计算下列函数的导数:

(1)$f(x)=\int_{\mathrm{sin}\ x}^{{\mathrm{cos}\ x}}\mathrm{e}^{(1+t)^{2}}\mathrm{d}t$

(2)$f(x)=\int_{x}^{x^{2}}\mathrm{e}^{-x^{2}u^{2}}\mathrm{d}u$

(3)$f(x)=\int_{a+x}^{b+x}\frac{\mathrm{sin}\ xt}{t}\mathrm{d}t$

(4)$f(u)=\int_{0}^{u}g(x+u,x-u)\mathrm{d}x$

(5)$f(t)=\int_{0}^{t^{2}}\mathrm{d}x\int_{x-t}^{x+t}\mathrm{sin}(x^{2}+y^{2}-t^{2})\mathrm{d}y$
\subsection*{3.利用对参数的微分法,计算积分}

这部分内容的特点是积分上下限都为有限值,假如\textbf{上限是}$+\infty$\textbf{才会去考虑是否一致收敛},如果上下限都是有限值,那么只要$f(x,u)$和$\frac{\partial f(x,u)}{\partial u}$ 连续,即可用这种方法计算复杂的积分.

\textbf{练习:}

利用对参数的微分法,计算下列积分:

(1)$\int_{0}^{\pi/2}\mathrm{ln}(a^{2}\mathrm{sin}^{2}x+b^{2}\mathrm{cos}^{2}x)\mathrm{d}x$

(2)$\int_{0}^{\pi/2}\mathrm{ln}\frac{1+a\mathrm{cos}\ x}{1-a\mathrm{cos}\ x}\frac{\mathrm{d}x}{\mathrm{cos}\ x},(|a|<1)$

(3)$\int_{0}^{\pi/2}\frac{\mathrm{arctan}(a\mathrm{tan}\ x)}{\mathrm{tan}\ x}\mathrm{d}x$,(注意讨论$a$的正负)

(4)$\int_{0}^{\pi}\mathrm{ln}(1+a\mathrm{cos}\ x)\mathrm{d}x,|a|<1$

(5)$\int_{0}^{1}\frac{x^{2}-x}{\mathrm{ln}\ x}\mathrm{d}x$

(6)$\int^{1}_{0}\frac{\mathrm{arctan}\ x}{x\sqrt{1-x^{2}}}\mathrm{d}x$

(7)$\int_{0}^{1}\frac{\mathrm{ln}(1+x)}{1+x^{2}}\mathrm{d}x$

(8)$\int_{0}^{a}\mathrm{arctan}\sqrt{\frac{a-x}{a+x}}\mathrm{d}x$

利用交换积分次序的方法,计算下列积分:

(9)$\int_{0}^{1}\mathrm{sin}(\mathrm{ln}\frac{1}{x})\frac{x^{b}-x^{a}}{\mathrm{ln}\ x}\mathrm{d}x$


\subsection*{4.一致收敛}


对于积分上限是$+\infty$的积分,假如交换积分和极限,或者积分号内求导,就必须加上条件是对参数一致收敛

例如$$\lim_{u\rightarrow u_{0}}\int_{a}^{+\infty}f(x,u)\mathrm{d}x=\int_{a}^{+\infty}\lim_{u\rightarrow u_{0}}f(x,u)\mathrm{d}x$$
成立的条件就不只有$f$在$[a,+\infty]\times[\alpha,\beta]$上连续了,还要加上$\int_{a}^{+\infty}f(x,u)\mathrm{d}x$在$[\alpha,\beta]$一致收敛.

所以,如果某个求极限的题需要用到极限和积分换序,并且积分上限是无穷,一定要先说明这个积分在极限点附近一致收敛.

同理,假如用到对参数求导后求积分这种方法时,如果积分上限是无穷,也需要先判断是不是在$[\alpha,\beta]$一致收敛,其中$[\alpha,\beta]$是参数需要求积分的区域.



证明一致收敛的常用方法有3个,分别是控制判别法,Abel,dirichlet

证明不一致收敛只能用cauchy.

控制判别法:假如存在$F(x)$对于任意的$u\in[\alpha,\beta]$,都有$|f(x,u)|<F(x)$,并且$\int_{a}^{+\infty}F(x)\mathrm{d}x$收敛,则$f(x,u)$一致收敛

dirichlet:$f(x,u)$对于任意的$u$都是单调的,并且当$x$趋于无穷时对于$u\in[\alpha,\beta]$一致的趋于0,$\int_{a}^{+\infty}g(x,u)\mathrm{d}x$对于$u\in[\alpha,\beta]$一致有界,可以推出$\int_{a}^{+\infty}f(x,u)g(x,u)\mathrm{d}x$一致收敛.

abel:$\int_{a}^{+\infty}f(x,u)\mathrm{d}x$对于$u\in[\alpha,\beta]$一致收敛,$g(x,u)$对于任意的$u$都是单调的,并且对于$u\in[\alpha,\beta]$一致有界,可以推出$\int_{a}^{+\infty}f(x,u)g(x,u)\mathrm{d}x$一致收敛.

\textbf{练习:}

证明下列积分在指定的区间一致收敛.

(1)$\int_{0}^{+\infty}\mathrm{e}^{-(1+x^{2})t}\mathrm{sin}\ t\mathrm{d}x,t\in[0,+\infty)$(控制判别法)(再提示:$|\mathrm{e}^{-(1+x^{2})t}\mathrm{sin}\ t|\leq t\mathrm{e}^{-x^{2}t}\leq \frac{t}{1+x^{2}t}\leq \frac{1}{x^{2}}$)



(2)$\int_{0}^{+\infty}\mathrm{e}^{-xu}\frac{\mathrm{sin}\ x}{x+u}\mathrm{d}x,u\in[0,+\infty)$(abel)

(3)$\int_{0}^{+\infty}\mathrm{e}^{-xu}\mathrm{sin}\ x\mathrm{d}x,0<u_{0}\leq u<+\infty$(控制判别法,abel)

(4)$\int_{-\infty}{+\infty}\frac{x^{2}\mathrm{cos}\ ux}{1+x^{4}}\mathrm{d}x,-\infty<u<+\infty$(控制判别法)

(5)$\int_{0}^{+\infty}\frac{\mathrm{d}x}{1+(x+u)^{2}},u\in[0,+\infty)$(控制判别法)

(6)$\int_{1}^{+\infty}\mathrm{e}^{-xu}\frac{\mathrm{cos}\ x}{\sqrt{x}}\mathrm{d}x,u\in[0,+\infty)$(abel)

对于证明不一致收敛,只能利用cauchy来证明,即证明对于\textbf{任意的}$A>0$,都存在一个$u_{0},A_{1}$,使得$$\left|\int_{A}^{A_{1}}f(x,u_{0})\mathrm{d}x\right|\leq \varepsilon_{0}$$

\textbf{例:}  证明:$$\int_{0}^{+\infty}\frac{x\mathrm{cos}\ ux}{a^{2}+x^{2}}\mathrm{d}x$$在$(0,+\infty)$不一致收敛.

\textbf{思路:}当$u$恒大于某个任意的正数$c$时,$|\int_{0}^{+\infty}\mathrm{cos}\ ux\mathrm{d}x|\leq \frac{2}{u}\leq\frac{2}{c}$对$u$一致有界,而$\frac{x}{a^{2}+x^{2}}$对于每个$u$都是单调的,并且对$u$一致趋于0,所以 由dirichlet可知原积分在$[c,+\infty)$一致收敛.

所以上述积分不一致收敛的原因就在于$u$\textbf{可以趋于0}.

由于$\int_{0}^{+\infty}\frac{x}{a^{2}+x^{2}}\mathrm{d}x$是发散的,所以对于任意的$A$,总存在$A_{1}$,使得

$$\left|\int_{A}^{A_{1}}\frac{x}{a^{2}+x^{2}}\mathrm{d}x\right|\geq \varepsilon_{0}$$

由于u是可以趋于0的,所以对于任意的$A$对应的$A_{1}$,总能找到足够小的$u$,使得$uA_{1}<\pi/4$,
所以对于任意的$x\in[A,A_{1}]$,都有$ux<\frac{\pi}{4}$,即$\mathrm{cos}\ ux>\frac{\sqrt{2}}{2}$

所以$$\left|\int_{A}^{A_{1}}\frac{x\mathrm{cos}\ ux}{a^{2}+x^{2}}\mathrm{d}x\right|>\frac{\sqrt{2}}{2}\left|\int_{A}^{A_{1}}\frac{x}{a^{2}+x^{2}}\mathrm{d}x\right|\geq\frac{\sqrt{2}}{2}\varepsilon_{0}$$

即不一致收敛.

\textbf{例:} 证明:$$\int_{0}^{+\infty}\sqrt{u}\mathrm{e}^{-ux^{2}}\mathrm{d}x$$在$[0,+\infty)$不一致收敛.

\textbf{提示:}对于任意的$A$,取$A_{1}=2A$,则作换元$t=\sqrt{u}x$,由于u可以趋于0,不妨取$u=1/A$,则$t\in[1,2]$
原积分化为$\int_{1}^{2}\mathrm{e}^{-t^{2}}\mathrm{d}t$,这个值是一个大于0的常数,所以不是一致收敛.

\subsection*{4.利用对参数求微分计算积分}

假如积分上限是无穷,则必须有一致收敛,才能对参数求导.

(1)计算积分$$\int_{0}^{+\infty}\frac{\mathrm{e}^{-ax}-\mathrm{e}^{-bx}}{x}\mathrm{sin}\ cx\mathrm{d}x,(a,b,c>0)$$

(2)利用已知的积分$\int_{0}^{+\infty}\mathrm{e}^{-x^{2}}\mathrm{d}x=\frac{\sqrt{\pi}}{2},\int_{0}^{+\infty}\frac{\mathrm{sin}\ x}{x}\mathrm{d}x=\frac{\pi}{2}$,计算下列积分

$$\int_{0}^{+\infty}\mathrm{e}^{-(2x^{2}+x+2)}\mathrm{d}x$$
$$\int_{0}^{+\infty}\frac{\mathrm{sin}\ x^{2}}{x}\mathrm{d}x$$
$$\int_{0}^{+\infty}\frac{\mathrm{sin}^{3}x}{x}\mathrm{d}x$$
\newpage
\subsection*{答案}
1.(1)1

1.(2)$\frac{8}{3}$

1.(3)$\frac{\pi}{16}$

2.(1)$-\mathrm{e}^{(1+\mathrm{cos}\ x)^{2}}\mathrm{sin}\ x-\mathrm{e}^{(1+\mathrm{sin}\ x)^{2}}\mathrm{cos}\ x$

2.(2)$-2x\int_{x}^{x^{2}}\mathrm{e}^{-x^{2}u^{2}}u^{2}\mathrm{d}u+2x\mathrm{e}^{-x^{6}}-\mathrm{e}^{-x^{4}}$

2.(3)$\frac{1}{x}(\mathrm{sin}\ x(b+x)-\mathrm{sin}\ x(a+x))+\frac{\mathrm{sin}(b+x)}{b+x}-\frac{\mathrm{sin}(a+x)}{a+x}$

2.(4)$\int_{0}^{u}(g_{1}'(x+u,x-u)-g_{2}'(x+u,x-u))\mathrm{d}x+g(2u,0)$

2.(5)$\int^{t^{2}}_{0}[-2t\int_{x-t}^{x+t}\mathrm{cos}(x^{2}+y^{2}-t^{2})\mathrm{d}y+\mathrm{sin}(x^{2}+(x+t)^{2}-t^{2})+\mathrm{sin}(x^{2}+(x-t)^{2}-t^{2})]\mathrm{d}x
+2t\int_{t^{2}-t}^{t^{2}+t}\mathrm{sin}\ y^{2}\mathrm{d}y$

3.(1)$\pi \mathrm{ln}\frac{a+b}{2}$

3.(2)$\mathrm{arcsin}\ a$

3.(3)$\pm\frac{\pi}{2}\mathrm{ln}(1+|a|)$,若$a>0$,取正号,若$a<0$,取负号.

3.(4)$2\pi \mathrm{ln}\mathrm{cos}\frac{\mathrm{arcsin}\ a}{2}$,或$\pi\mathrm{ln}\frac{1+\sqrt{1-a^{2}}}{2}$

3.(5)$\mathrm{ln}\ \frac{3}{2}$

3.(6)$\frac{\pi}{2}\mathrm{ln}(1+\sqrt{2})$

3.(7)$\frac{\pi\mathrm{ln}\ 2}{8}$

3.(8)$\frac{1}{2}a$

3.(9)$\mathrm{arctan}(1+b)-\mathrm{arctan}(1+a)$

4.(1)$\mathrm{arctan}\ \frac{b}{c}-\mathrm{arctan}\ \frac{a}{c}$

4.(2)$\frac{\sqrt{2\pi}}{2}\mathrm{e}^{-15/8}$

$\frac{\pi}{4}$

$\frac{\pi}{4}$
\end{document}
