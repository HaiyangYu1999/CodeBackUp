\documentclass{ctexart}
\pagestyle{plain}
\usepackage{amsmath}
\begin{document}
\begin{flalign}
\int_{0}^{+\infty}x\mathrm{e}^{-\alpha x^{2}}\ \mathrm{sin}\ bx\ \mathrm{d}x&=-\frac{1}{2\alpha}\int_{0}^{+\infty}\mathrm{sin}\ bx\ \mathrm{d}\mathrm{e}^{-\alpha x^{2}}&\nonumber\\
&=-\frac{1}{2\alpha}[\mathrm{e}^{-\alpha x^{2}}\mathrm{sin}\ bx]^{+\infty}_{0}+\frac{1}{2\alpha}\int_{0}^{+\infty}\mathrm{e}^{-\alpha x^{2}}\ \mathrm{d}(\mathrm{sin}\ bx)&\nonumber\\
&=\frac{b}{2\alpha}\int_{0}^{+\infty}\mathrm{e}^{-\alpha x^{2}}\ \mathrm{cos}\ bx\ \mathrm{d}x\nonumber
\end{flalign} 
注意到$\int_{0}^{+\infty}\mathrm{e}^{-\alpha x^{2}}\ \mathrm{cos}\ bx\ \mathrm{d}x$对$b$的导数是$-\int_{0}^{+\infty}x\mathrm{e}^{-\alpha x^{2}}\ \mathrm{sin}\ bx\ \mathrm{d}x$,令$I(b)=\int_{0}^{+\infty}\mathrm{e}^{-\alpha x^{2}}\ \mathrm{cos}\ bx\ \mathrm{d}x$,这样由上面的推导就有$$I'(b)=-\frac{b}{2\alpha}I(b)$$
即$$\frac{\mathrm{d}I}{\mathrm{d}b}=-\frac{bI}{2\alpha}$$
$$\frac{\mathrm{d}I}{I}=-\frac{b\ \mathrm{d}b}{2\alpha}$$
两边同时对积分得
$$\mathrm{ln}\ I=-\frac{b^{2}}{4\alpha}+c$$
即$$I=c\mathrm{e}^{-\frac{b^{2}}{4\alpha}}$$
又因为$b=0$时,
$$I(0)=c=\int_{0}^{+\infty}\mathrm{e}^{-\alpha x^{2}}\ \mathrm{d}x=\frac{\sqrt{\pi}}{2\sqrt{\alpha}}$$
所以
$$I(b)=\frac{\sqrt{\pi}}{2\sqrt{\alpha}}\mathrm{e}^{-\frac{b^{2}}{4\alpha}}$$
原积分$=\frac{b}{2\alpha}I(b)=\frac{\sqrt{\pi}b}{4\alpha\sqrt{\alpha}}\mathrm{e}^{-\frac{b^{2}}{4\alpha}}$
\end{document} 