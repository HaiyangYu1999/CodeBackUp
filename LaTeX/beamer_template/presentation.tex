\documentclass[10pt]{beamer}
\usepackage{ctex}
\usepackage{color}
\usepackage{times}
\usepackage{subfigure}
\usepackage{bm}
\usepackage{minted}
\usefonttheme[onlymath]{serif}
\CTEXoptions[today=old]
\usetheme{CambridgeUS}
\usepackage{amsmath}
\setmainfont{Times New Roman}

\begin{document}

\title{单词P58-60}
\author{{\fangsong 于海洋}}
\institute{{\fangsong 数学学院}}
\frame{\titlepage}
\begin{frame}{bias}


\begin{itemize}
\item{bias   n. {\fangsong 偏差; 偏见; 偏爱;}}
\item{The {\color{magenta} bias} of an estimator is the difference between an estimator's expected value and the true value of the parameter being estimated. }

{估计量的{\color{magenta}偏差}是估计量的期望值和被估计参数的真值之间的差。}

\item{Observer {\color{magenta} bias} arises when the researcher subconsciously influences the experiment due to cognitive {\color{magenta} bias} where judgement may alter how an experiment is carried out.}

{当研究者由于认知{\color{magenta}偏差}而下意识地影响实验时,会产生观察者{\color{magenta}偏差},因为判断可能会改变实验的执行方式。}
\end{itemize}

\end{frame}


\begin{frame}{biased, unbiased}
\begin{itemize}
\item{biased {\fangsong 有偏的}, unbiased{\fangsong 无偏的}}
\item{biased estimator {\fangsong 有偏估计},  unbiased estimator {\fangsong 无偏估计}}
\end{itemize}

\begin{small}
\begin{itemize}

\item{Supposing $\hat{\theta}$  is an estimator of parameter $\theta$, define $B(\hat{\theta})=E(\hat{\theta})-\theta$.
The estimator $\hat{\theta}$
 is an unbiased estimator of $\theta$ if and only if $B(\hat{\theta})=0$.
}
\item{These estimators are unbiased under normality and less biased than estimator under non normality.}
\end{itemize}
\end{small}
\end{frame}


\begin{frame}{big, bigger, biggest, biggish, bigness}
\begin{itemize}
\item{$a<b$, $a$ is less than $b$.}
\item{$a>b$, $a$ is greater than $b$.}
\item{$a\leq b$, $a$ is not greater than $b$.}
\item{$a\geq b$, $a$ is not less than $b$.}
\item{$a=b$, $a$ is equal to $b$.}
\end{itemize}


\end{frame}

\begin{frame}{binary}
\begin{itemize}
\item{binary {\fangsong 二进制的}, binary number {\fangsong 二进制数}}
\item{decimal {\fangsong 十进制的}, octal {\fangsong 八进制的}, hexadecimal {\fangsong 十六进制的}}
\begin{footnotesize}
\item{The modern {\color{magenta}binary} number system was studied in Europe in the 16th and 17th centuries.

现代二进制数系统是16世纪和17世纪在欧洲研究的。}
\end{footnotesize}
\end{itemize}
\end{frame}

\begin{frame}{binary}
\begin{itemize}
\item{binary {\fangsong 二元的}, binary operator {\fangsong 二元运算符}, binary function {\fangsong 二元函数}}
\item{unary operator {\fangsong 一元运算符}, ternary operator {\fangsong 三元运算符}}
\begin{footnotesize}
\item{Typical examples of {\color{magenta}binary} operations are the addition (+) and multiplication (×) of numbers and matrices as well as composition of functions on a single set.

二元运算的典型例子是数字和矩阵的加(+)和乘(×),以及单个集合上函数的复合。}
\item{A function $f$ is {\color{magenta}binary} if there exists sets $X, Y, Z$
 such that $$f: X\times Y\rightarrow Z$$
where $X\times Y$
 is the Cartesian product of $X$ and $Y$.
 }
\end{footnotesize}
\end{itemize}
\end{frame}

\begin{frame}{binary}
\begin{itemize}
\item{binary {\fangsong 二分的,二叉的}}
\item{binary search {\fangsong 二分查找}, bisection method {\fangsong 二分法}, binary tree {\fangsong 二叉树}, binary heap {\fangsong 二叉堆}}
\begin{footnotesize}
\item{A {\color{magenta}binary} tree is a tree data structure in which each node has at most two children.

二叉树是一种树数据结构,其中每个节点最多有两个子节点。}
\item{{\color{magenta}Binary} heaps are a common way of implementing priority queues.

二叉堆是实现优先级队列的常用方法。}
\end{footnotesize}
\end{itemize}
\end{frame}
\begin{frame}{bind}
\begin{itemize}
\item{bind v.{\fangsong 结合}}
\begin{footnotesize}
\item{As in the case of vector spaces, we stipulate that multiplication {\color{magenta}binds} stronger than addition, and we usually write $\lambda v$ for $\lambda·v$.

在向量空间的情况下,我们规定乘法比加法结合更强,我们通常将$\lambda·v$写作为$\lambda v$。 }
\item{Molecular {\color{magenta}binding} is an attractive interaction between two molecules that results in a stable association in which the molecules are in close proximity to each other.

分子结合是两个分子之间的一种吸引的相互作用,它导致一种稳定的结合,其中分子彼此非常接近。 }
\end{footnotesize}
\end{itemize}
\end{frame}
\begin{frame}{bound}
\begin{itemize}
\item{bound {\fangsong 限制, 约束}  bound variable {\fangsong 约束变量}, free variable {\fangsong 自由变量}}
\begin{footnotesize}
\item{A {\color{magenta}bound} variable is a variable that was previously free, but has been bound to a specific value or set of values called domain of discourse or universe.

约束变量是一个之前是自由的,但被绑定到一个特定的值或一组称为论域的值上的变量。}
\end{footnotesize}
\end{itemize}

\begin{itemize}
\item{upper bound {\fangsong 上界}, supremum {\fangsong 上确界}, maximum {\fangsong 最大值}}
\item{lower bound {\fangsong 下界}, infimum  {\fangsong 下确界}, minimum {\fangsong 最小值}}
\end{itemize}
\end{frame}


\end{document}
