\documentclass{article}
\usepackage{graphicx}
\usepackage{amssymb}
\usepackage{subfigure}
\usepackage{geometry}
\usepackage{bm}
\usepackage{fancyhdr}
\usepackage{minted}
\geometry{a4paper,left=3cm,right=3cm,top=3cm,bottom=3cm}
\usepackage{amsmath}
\pagestyle{plain}
\title{STT886 Homework\#5}
\author{Haiyang Yu}
\begin{document}
\maketitle
\subsection*{1.}
$$P_{T}=
 \left[
 \begin{matrix}
   0.4 & 0.2 & 0.1 \\
   0.1 & 0.5 & 0.2 \\
   0.3 & 0.4 & 0.2
  \end{matrix}
  \right]
$$
$$S=(I-P_{T})^{-1}=
\left[
\begin{matrix}
2.20689655 &1.37931034& 0.62068966\\
0.96551724 &3.10344828 &0.89655172\\
1.31034483 &2.06896552 &1.93103448
\end{matrix}
\right]
$$
Thus, $(s_{13},s_{23},s_{33})^{\mathrm{T}}=(1.31034483, 2.06896552, 1.93103448)^{\mathrm{T}}$.

$$f_{ij}=\frac{s_{ij}-\delta_{ij}}{s_{jj}}$$

$(f_{13},f_{23},f_{33})^{\mathrm{T}}=(0.6785714307908163, 1.071428574387755, 0.48214285640306126)$
\subsection*{2.}
Because we have $E(X_{n})=\mu^{n}$, The number of individuals ever exist is $\sum_{n=0}^{+\infty}X_{n}$. Thus, 
\begin{align*}
E(\mathrm{the\ number\ of\ individuals\ ever\ exist})&=E(\sum_{n=0}^{+\infty}X_{n})\\
&=\sum_{n=0}^{+\infty}\mu^{n}\\
&=\frac{1}{1-\mu}
\end{align*}
If $X_{0}=n$, we have $E(X_{i})=n\mu^{i}$, The number of individuals ever exist is $\sum_{i=0}^{+\infty}nX_{i}$. Thus,
\begin{align*}
E(\mathrm{the\ number\ of\ individuals\ ever\ exist})&=E(\sum_{i=0}^{+\infty}X_{i})\\
&=\sum_{i=0}^{+\infty}n\mu^{i}\\
&=\frac{n}{1-\mu}
\end{align*}
\subsection*{3.}
$$\pi_{0}=\frac{1}{4}\pi_{0}+\frac{3}{4}\pi_{0}^{2}\Rightarrow \pi_{0}=\frac{1}{3}$$
$$\pi_{0}=\frac{1}{4}+\frac{1}{2}\pi_{0}+\frac{1}{4}\pi_{0}^{2}\Rightarrow \pi_{0}=1$$
$$\pi_{0}=\frac{1}{6}+\frac{1}{2}\pi_{0}+\frac{1}{3}\pi_{0}^{3}\Rightarrow \pi_{0}=\frac{\sqrt{3}-1}{2}$$
\subsection*{4.}
\subsubsection*{(a)} 
Yes. Because the number of white balls only depends on the last time's number of white balls.
\subsubsection*{(b)}
Class:$\{0,1,2,\cdots,N\}$. Aperiodic. Recurrent.
\subsubsection*{(c)}
$$P_{i,i+1}=p\frac{N-i}{N},\ i=0,1,\cdots,N-1$$
$$P_{i,i}=p\frac{i}{N}+(1-p)\frac{N-i}{N},\ i=0,1,\cdots,N$$
$$P_{i,i-1}=(1-p)\frac{i}{N},\ i=1,2,\cdots,N$$
\subsubsection*{(d)}
$N=2$. Let
$$\bm{\pi}^{\mathrm{T}}\left[
\begin{matrix}
1-p&p&0\\
(1-p)/2&p/2+(1-p)/2&p/2\\
0&1-p&p
\end{matrix}
\right]
=\bm{\pi}^{\mathrm{T}}$$
$$\sum_{i=0}^{2}\pi_{i}=1$$
Thus,$$\bm{\pi}^{\mathrm{T}}=((1-p)^{2},2p(1-p),p^{2}),$$ which means $$\pi_{i}=\binom{N}{i}p^{i}(1-p)^{N-i},\ N=2$$
\subsubsection*{(e)}
So we guess $$\pi_{i}=\binom{N}{i}p^{i}(1-p)^{N-i}.$$
\subsubsection*{(f)}
When $1\leq i\leq N$, we have
\begin{footnotesize}
\begin{align*}
\sum_{j=1}^{N}\pi_{j}P_{j,i}=&P_{i-1,i}\cdot\pi_{i-1}+P_{i,i}\cdot\pi_{i}+P_{i+1,i}\cdot\pi_{i+1}\\
=&p\frac{N-i+1}{N}\binom{N}{i-1}p^{i-1}(1-p)^{N-i+1}+\left(p\frac{i}{N}+(1-p)\frac{N-i}{N}\right)\binom{N}{i}p^{i}(1-p)^{N-i}\\
&+(1-p)\frac{i+1}{N}\binom{N}{i+1}p^{i+1}(1-p)^{N-i-1}\\
=&p\frac{N-i+1}{N}\frac{N!}{(N-i+1)!(i-1)!}p^{i-1}(1-p)^{N-i+1}+\left(p\frac{i}{N}+(1-p)\frac{N-i}{N}\right)\frac{N!}{(N-i)!i!}p^{i}(1-p)^{N-i}\\
&+(1-p)\frac{i+1}{N}\frac{N!}{(N-i-1)!(i+1)!}p^{i+1}(1-p)^{N-i-1}\\
=&(1-p)\frac{i}{N}\frac{N!}{(N-i)!i!}p^{i}(1-p)^{N-i}+\left(p\frac{i}{N}+(1-p)\frac{N-i}{N}\right)\frac{N!}{(N-i)!i!}p^{i}(1-p)^{N-i}\\
&+p\frac{N-i}{N}\frac{N!}{(N-i)!i!}p^{i}(1-p)^{N-i}\\
=&\binom{N}{i}p^{i}(1-p)^{N-i}\\
=&\pi_{i}
\end{align*}
\end{footnotesize}
And it is obvious that $\pi_{0}=P_{00}\cdot\pi_{0}+P_{10}\cdot\pi_{1}$, $\pi_{N}=P_{N-1,N}\cdot\pi_{N-1}+P_{NN}\cdot\pi_{N}$.

Thus, $\bm{\pi}^{\mathrm{T}}=\bm{\pi}^{\mathrm{T}}P$, which indicates $$\pi_{i}=\binom{N}{i}p^{i}(1-p)^{N-i}.$$
\subsubsection*{(g)}
Suppose it takes $K_{j}$ times to turn from $j$ white to $j+1$ white.
So$$P(K_{j}=k)=\left(\frac{j}{N}\right)^{k-1}\frac{N-j}{N}$$
And
\begin{align*}
E(K_{j})&=\sum_{k=1}^{\infty}k\left(\frac{j}{N}\right)^{k-1}\frac{N-j}{N}\\
&=\frac{N-j}{N}\sum_{k=1}^{\infty}\left[\frac{\mathrm{d}}{\mathrm{d}x}x^{k}\right]_{x=\frac{j}{N}}\\
&=\frac{N-j}{N}\left[\frac{\mathrm{d}}{\mathrm{d}x}\sum_{k=1}^{\infty}x^{k}\right]_{x=\frac{j}{N}}\\
&=\frac{N-j}{N}\left[\frac{\mathrm{d}}{\mathrm{d}x}\left(\frac{x}{1-x}\right)\right]_{x=\frac{j}{N}}\\
&=\frac{N-j}{N}\left[\frac{1}{(1-x)^{2}}\right]_{x=\frac{j}{N}}\\
&=\frac{N-j}{N}\frac{1}{(1-\frac{j}{N})^{2}}\\
&=\frac{N}{N-j}
\end{align*}
So the times are
\begin{align*}
E(\sum_{j=i}^{N-1}K_{j})&=\sum_{j=i}^{N-1}E(K_{j})\\
&=\sum_{j=i}^{N-1}\frac{N}{N-j}\\
&=\sum_{j=1}^{N-i}\frac{N}{j}
\end{align*}

\end{document}
